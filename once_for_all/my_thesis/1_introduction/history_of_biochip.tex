The field of microfluidics has four parents: molecular analysis, biodefence, molecular biology and microelectronics. 
First came analysis. The distant origins of microfluidics lie in microanalytical methods — gas-phase chromatography (GPC), 
high-pressure liquid chromatography (HPLC) and capillary electrophoresis (CE) — which, in capillary format, revolutionized chemical analysis. These methods (combined with the power of the laser in optical detection) made it possible to simultaneously achieve high sensitivity and high r
esolution using very small amounts of sample. With the successes of these microanalytical methods, it seemed obvious to develop new, more compact and more versatile formats for them, and to look for other applications of microscale methods in chemistry and biochemistry. A second, different, motiva


A second, different, motivation for the development of microfluidic systems came with the realization — after the end of the cold war — that chemical and biological weapons posed major military and terrorist threats. To counter these threats, the Defense Advanced Research Projects Agency (DARPA)
 of the US Department of Defense supported a series of programmes in the 1990s aimed at developing field-deployable microfluidic systems designed to serve as detectors for chemical and biological threats. These programmes were the main stimulus for the rapid growth of academic microfluidic technology.

The third motivational force came from the field of molecular biology. The explosion of genomics in the 1980s, followed by the advent of other areas of microanalysis related to molecular biologies, such as high-throughput DNA sequencing, required analytical methods with much greater throughput, 
and higher sensitivity and resolution than had previously been contemplated in biology. Microfluidics offered approaches to overcome these problems. The fourth contribution was from microe


The original hope of microfluidics was that photolithography and associated 
technologies that had been so successful in silicon microelectronics,
[The origins and the future of microfluidic]

Microfluidics have come a long way since the seminal paper by Andreas Manz in 
1990 (Manz,   1990).   The   paper   envisioned   an   integrated,
   automated   platform   for performing a range of analysis steps. 
The advances since then have brought us closer to  the  vision  of  a  sample-in-answer-out  platform  (Jenkins  &  Mansfield,  2013;  Lee, 2013). At the turn of the century, the introduction of PDMS and soft lithography gave a  major  boost  to  the  field  (Duffy,  et  al.,  1998;  Unger,  et  al.,  
2000).  These  simple, inexpensive, and rapid microfabrication techniques have enabled researches to apply microfluidics  to  a  wide  range  of  areas  like  catalysis,  molecular  point-of-care  (POC) diagnostics remains the primary application area of microfluidics (Jayamohan, et al., 2013)  and  it  accounts  for  the  largest  share  of  the  microfluidics  market  (Yetisen  & Volpatti, 2014).
[Advances in Microfluidics and Lab-on-a-Chip Technologies


While  PCR  has  long  been  used  for  mutation  detection,  a  highly  sensitive  version  of PCR,  digital  PCR,  has  been  gaining  a  significant  following  both  commercially  and scientifically. 

Microfluidics   can   be   broadly   defined   as   systems   leveraging   micrometer  scale channels,  to  manipulate  and  process  low  volum]
this will result in broadly defined systems.
