
% this file is called up by thesis.tex
% content in this file will be fed into the main document

%: ----------------------- introduction file header -----------------------
\chapter{Introduction}

% the code below specifies where the figures are stored
\ifpdf
    \graphicspath{{1_introduction/figures/PNG/}{1_introduction/figures/PDF/}{1_introduction/figures/}}
\else
    \graphicspath{{1_introduction/figures/EPS/}{1_introduction/figures/}}
\fi

% ----------------------------------------------------------------------
%: ----------------------- introduction content ----------------------- 
% ----------------------------------------------------------------------



%: ----------------------- HELP: latex document organisation
% the commands below help you to subdivide and organise your thesis
%    \chapter{}       = level 1, top level
%    \section{}       = level 2
%    \subsection{}    = level 3
%    \subsubsection{} = level 4
% note that everything after the percentage sign is hidden from output



\section{The rise of the chip} % section headings are printed smaller than chapter names
% intro

The field of microfluidics has four parents: molecular analysis, biodefence, molecular biology and microelectronics. 
First came analysis. The distant origins of microfluidics lie in microanalytical methods — gas-phase chromatography (GPC), 
high-pressure liquid chromatography (HPLC) and capillary electrophoresis (CE) — which, in capillary format, revolutionized chemical analysis. These methods (combined with the power of the laser in optical detection) made it possible to simultaneously achieve high sensitivity and high resolution using very small amounts of sample. With the successes of these microanalytical methods, it seemed obvious to develop new, more compact and more versatile formats for them, and to look for other applications of microscale methods in chemistry and biochemistry. A second, different, motiva

The first key area that inspired the field of microfluidics is molecular analysis. 

A second, different, motivation for the development of microfluidic systems came with the realization — after the end of the cold war — that chemical and biological weapons posed major military and terrorist threats. To counter these threats, the Defense Advanced Research Projects Agency (DARPA) of the US Department of Defense supported a series of programmes in the 1990s aimed at developing field-deployable microfluidic systems designed to serve as detectors for chemical and biological threats. These programmes were the main stimulus for the rapid growth of academic microfluidic technology.

The third motivational force came from the field of molecular biology. The explosion of genomics in the 1980s, followed by the advent of other areas of microanalysis related to molecular biologies, such as high-throughput DNA sequencing, required analytical methods with much greater throughput, and higher sensitivity and resolution than had previously been contemplated in biology. Microfluidics offered approaches to overcome these problems. The fourth contribution was from microe


The original hope of microfluidics was that photolithography and associated technologies that had been so successful in silicon microelectronics,
[The origins and the future of microfluidic]


Microfluidic biochips are revolutionizing the traditional biochemical experiment 
flow with their high execution efficiency and 
miniaturized fluid manipulation \cite{JEMP08, EVNR03,JMSQ07}. 
Devices are built 
%from microchannels and valves 
in such a chip to execute specific operations, such as mixing and detection.
Fluid samples are transported through microchannels between devices to carry
out the protocol of a bioassay. 
All these functions are performed at the nanoliter level and
controlled by a microcontroller without human intervention. 
The efficiency and reliability of such miniaturized and automated chips 
endow them with a great potential to improve human life significantly, 
and the research to bridge them with real-world applications
%bring them from prototypes in laboratories to industry production 
is key to their success.   

\begin{figure}[htpb]
    {      
    \centering
    % \input{Fig/valve_mixer_chip.pdf_tex}
    \input{1_introduction/figures/valve_mixer_chip.pdf_tex}
    %\includegraphics{Fig/valve_mixer_chip.png}
    \caption{Components and structure of flow-based biochips. 
    (a) Valves constructed at intersections of flow/control channels \cite{JMSQ07}. 
    (b) Mixer \cite{JMSQ07}. (c) A part of a biochip
    containing a mixer surrounded by a transportation channel network 
    \cite{ESWD13}.}
    \label{fig:valve_mixer_chip}
    }
    \end{figure}

A flow-based microfluidic biochip is constructed from basic components such as
%basic components, namely, 
microchannels and microvalves, henceforth named as channels and valves for
simplicity.
Flow channels are used to transport reaction samples and reagents
between different locations. Above flow channels, control 
channels are built to 
conduct air pressure to intersections of flow channels and control channels 
to form 
valves, as illustrated in \figname~\ref{fig:valve_mixer_chip}(a),
where three valves are constructed at the intersections.
These channels are built from elastic materials, so that
air pressure in a control channel can block the movement of fluid sample
by squeezing the flow channel downwards.
Conversely, if the pressure in the control channel is
released, the fluid sample can resume its movement. 
%Functionally, valves are thus formed at intersections of
%flow channels and control channels, and the open/close states
%of valves are controlled by the air pressure fed into the 
%control channels. 
Since the channel width has been miniaturized
down to 50 um \cite{Studer04} thanks to the advance of manufacturing
technology, a huge number of channels and valves can
already be integrated
into a single biochip to perform large-scale experiments and diagnoses.

With valves as basic controlling components, complex devices 
can be constructed. For example, mixers can be built using
channels and valves to execute mixing operations, which are very
common in biochemical applications. The structure of a mixer is shown
in \figname~\ref{fig:valve_mixer_chip}(b),
where the three valves at the bottom are actuated alternately by 
applying and releasing air pressure in the control
channels 
%to form a circular flow around the device 
to mix fluid samples and reagents by peristalsis.
The execution of a mixing operation in a biochip is demonstrated
in a video \cite{mixing_store}. 
After the mixing operation is completed, the resulting fluid sample 
can be stored in a dedicated storage unit temporarily. %in the biochip.

In a biochip, devices executing specific operations, e.g., mixing and heating,
are connected by channels so that intermediate reaction results %fluid samples 
can be
transported between devices for processing. All these operations and
transportation are controlled
%by valves, whose open/close states are regulated
by a microcontroller, which 
%. To execute a biochemical assay, the microcontroller 
issues instructions in a given order to actuate valves 
to move fluid samples 
%between devices 
and execute operations.
%of valve actuation to move fluid samples
%to different devices to process them. 
%The photograph of a complete biochip is
%shown 
\figname~\ref{fig:valve_mixer_chip}(c) shows a mixer (reaction loop) surrounded
by flow channels (green), control channels (yellow and red) and valves
(yellow and red blocks).
%formed 
%at their
%At the 
%intersections.
%of flow and control channels, 
%used to direct fluid transportation. 
These channels and valves together form a network similar to the
road transportation system. If fluid channels should cross, %four
valves are built %at the crossing point 
to form a switch, as shown 
in \figname~\ref{fig:valve_mixer_chip}(c).
At any moment, only two out of the four valves should be opened to 
direct fluid transportation; 
%form a flow path; 
the other two valves must be closed to avoid fluid
contamination. Consequently, the role of the valves at the intersection
of %two 
flow channels is similar to that of the traffic lights in the road
transportation system, while the open/closed states of the valves are
controlled by a microcontroller according to the protocol of the application.
%and many valves are used to coordinate fluid
%transportation and assay execution. 
The mixer and the channel network
%The size of the chip in 
in \figname~\ref{fig:valve_mixer_chip}(c)
are implemented into a biochip of the size comparable to that of a coin 
as shown %in the photo 
at the upper left
corner, %of \figname~\ref{fig:valve_mixer_chip}(c), 
demonstrating the miniaturized integration of 
microfluidic biochips. 


\begin{figure}
    {
    \vskip -5pt
    \figurefontsize
    \centering
    \input{1_introduction/figures/sequencing_graph.pdf_tex}
    \caption{Sequencing graph of a bioassay.}
    \label{fig:sequencing_graph}
    }
    \end{figure}

    In a biochip, the open/closed states of valves and the transportation of
fluid samples are determined according to the biochemical application executed by the
biochip. A biochemical application, or \textit{bioassay} henceforth,  
%A biochemical application, or a bioassay, 
is usually described with
%describes the operations 
%executed by a biochip and their dependency with 
a \textit{sequencing graph} $\mathcal{G}=(\mathcal{O},\mathcal{E})$, such as in
\figname~\ref{fig:sequencing_graph}, where $\mathcal{O}$ is
the set of nodes %representing operations,
and $\mathcal{E}$ is the set of edges. %in the graph specifying the dependency
%relation between operations. 
A node $O_i \in \mathcal{O}$ in the sequencing graph
represents an operation, whose type $\tau_i$ and duration $u_i$ are specified by the user.
The type $\tau_i$ of the operation, e.g., mix, heat and filter, is predefined by the application. 
To execute an operation, the corresponding device must be built in the chip
and the operation must be assigned to this device.
An edge $e_{ij}\in \mathcal{E}$ from $O_i$ to $O_j$
in the sequencing graph specifies that $O_i$ must be executed before $O_j$ and the 
result of $O_i$ is the input of $O_j$. If $O_i$ and $O_j$ are executed by
different devices, the required fluid transportation must be performed by the channel
network between devices. 

Biochips have a huge advantage over the traditional manual experiment
flow, where operations performed by humans are error-prone 
and inaccurate.  Any inadvertent mistake in this manual process 
might ruin a complex experiment that may
last for several days. In a biochip, the volumes of fluid samples and reagents are
controlled accurately and fluid samples are moved to target devices reliably,
%because the whole experiment is controlled 
%all of which are managed
all of which are managed
by a microcontroller exactly following a
given protocol.
In addition, the miniaturized size of biochips makes them
extremely portable, so that a complex lab can be integrated into a single chip
and carried conveniently to perform on-the-spot tests to counter acute 
disease outbreaks 
such as the devastating Ebola virus disease a few years ago.
Furthermore, reactions with fluid samples and reagents of tiny volumes
take less time to complete than those with large volumes in tubes and
in the traditional experiment flow, so that biochips are also more
responsive in dealing with urgent situations.
Moreover, %Economically, 
biochips save precious reagents by performing operations at
nanoliter level. %The required 
%Such reagents may be exorbitantly expensive. 
For example, RNase inhibitor, a polyclonal antibody
commonly used in reverse transcription polymerase chain reaction, cost 600 euros per milliliter in December 2014 
\cite{RNasePrice}. 

The miniaturization of microfluidic biochips also has the potential of
large-scale
system integration. Already in 2008, a biochip array with 25K valves was
accomplished \cite{JMPK08}, and recent advances in manufacturing technologies have 
led to
%enabled
a valve density of 1 million per cm$^2$ %\si{\square\centi\metre} 
\cite{C2LC40258K}. 
A system integration of this scale
enables long-aspired exhaustive diagnoses in identifying the illness 
of a patient by testing pathological samples with thousands of reagents 
simultaneously. This breakthrough will not only reduce
the inaccuracy in medical diagnoses, where individual expertise and experience
of doctors play an important role, but also change the current 
guess-then-test model of medical treatment. 
In addition, such exhaustive diagnoses can be performed in
small health-care centers routinely, 
due to the
tremendously miniaturized chip size and lowered cost. 
With this exhaustive diagnosis model, illnesses can be detected at a very early stage and 
treatment cost can be reduced significantly as well.

Owing to their efficiency and cost-effectiveness, microfluidic biochips are
reshaping many fields such as pharmacy, biotechnology and health care.
%in academic research as well as in applications.
In recent years, genomic bioassay protocols, such as nucleic-acid
isolation, DNA purification and DNA sequencing, have been successfully
demonstrated with microfluidic biochips. In addition, this technology has 
attracted a lot of commercial attention, e.g., from Illumina \cite{illum}, 
a market leader in DNA sequencing. 
%Furthermore, 
Accordingly, 
%%Recently,
%Frost \& Sullivan has estimated that the European lab-on-chip and microfluidics market 
%reaches about 1.62 billion USD even at the early growth phase of this
%technology. Correspondingly, 
the International Technology Roadmap
for Semiconductors (ITRS) 2015 \cite{itrs}
has  
%listed medical applications as one of application drivers and 
recognized the importance of microfluidic devices as 
having a rapid growth in the next several years. 

%Due to the importance of microfluidic biochips to human life and the huge
%potential market, research on microfluidic biochips has been growing
%continuously recently. Scientific papers have been published, e.g.,
%%in IEEE Trans. on CAD and in ACM J. on Emerg.  Tech.
%in IEEE Transactions on Computer-Aided Design of Integrated Circuits and
%Systems and ACM Journal on Emerging Technologies in Computing.
%Several renowned universities, for instance, CMU and Duke, have initiated research
%projects on design automation for microfluidic biochips. 
%Accordingly,
%at the Technical University of Munich,  with the support of the Institute for
%Advanced Study, funded by the German Excellence Initiative and
%the European Union Seventh Framework Programme, we have also established a
%research group focusing on design automation and optimization of microfluidic
%biochips. %at micro and macro level.


%\vskip 8pt
\textit{\underline{Synthesis of microfluidic biochips  using
computer algorithms}}


In a biochip, 
%it is very rare to assign each operation to a dedicated device 
%for cost reasons.
%Instead, 
operations are executed by a given number of devices with time
multiplexing,
%. Consequently, the execution of operations is 
described 
as
a schedule. For example, an execution %of the operations 
of the bioassay
illustrated in \figname~\ref{fig:sequencing_graph} is shown 
in \figname~\ref{fig:biochip_synthesis}(a), 
where two mixers, one heater, one filter, and one detector are available.
%This schedule, however, can still be improved by executing 
%$O_3$, $O_4$ and $O_6$ before $O_1$ and $O_2$ to produce fluid samples
%processed by $O_8$ and $O_9$ earlier. 
%Since the latter two operations use heater and filter instead of mixers, an
%early result from $O_6$ increases the parallelism of operation execution
%and thus shortens the overall execution time of the bioassay. 
%In the last step of synthesis, 
According to the schedule, 


\begin{figure}
    {
    \vskip -3pt
    \figurefontsize
    \centering
    \input{1_introduction/figures/biochip_synthesis.pdf_tex}
    \caption{Synthesis of microfluidic biochips.
    (a) Scheduling. (b) Physical design.}
    \label{fig:biochip_synthesis}
    }
    \end{figure}

    the layout of a biochip, 
    including the locations of devices and the transportation channels between them,
    can be determined to generate a physical design, 
    as shown in \figname~\ref{fig:biochip_synthesis}(b), where the devices
    are connected by a channel network controlled by valves.
    
    The synthesis process above demonstrates that the schedule of operations of a
    bioassay determines the overall execution time. 
    %In addition, the communication
    %between devices also affects the channel network for fluid transportation.
    In addition, the fluid transportation between devices 
    affects the structure of the channel network.
    Consequently, a holistic design automation flow is required to bridge the
    low-level components introduced by the microfluidics community with high-level
    real-world applications. In each step of this design automation flow, various design
    objectives should be optimized to achieve an efficient architecture for the
    biochip.
    
    The synthesis flow of biochips
    %shown in \figname~\ref{fig:biochip_synthesis} 
    is similar
    to the synthesis flow for integrated circuits \cite{Micheli94}. Therefore,
    researchers in the electronic design automation community have started to expand
    into this area in recent years \cite{ChakrabartyFZ10,PopAC15}. 
    However, these research efforts 
    %on flow-based biochips is 
    are still in an early stage and many unique characteristics of microfluidic
    biochips have still not been considered. %to date. 
    %problems need to be solved soon to pave the way for industry production
    %of biochips. 
    %are open. %still unsolved.
    
    \vskip 8pt
    \textit{\underline{Flow-based microfluidic architectures: the electronic view}}
    
    In the microfluidics community, researchers are focusing on developing new
    technologies and new structures to build fundamental components and devices,
    such as valves and pumps
    \cite{Unger113,mathies2010multiplexed}.
    Prototype microfluidic biochips are also built very often %, but usually 
    %for the purpose of 
    to demonstrate the function and performance of new
    components and new devices.
    %Besides new devices, 
    Another major focus of the microfluidics community is to
    increase the integration density of basic components. With the advance in MEMS
    technology, a large number of components such as valves can now be built in a
    single biochip \cite{C2LC40258K}. 
    Unfortunately, the abundant available resources 
    %from the advance in the microfluidc community
    have mostly been left unexplored, because end users cannot use them 
    without a system layer that presents an interface for user applications,
    similar to the scenario that an operating system is missing 
    for computer users. On the other hand, 
    %the lack
    %of applications has disheartened the microfluidics research community and
    %consequently their effort 
    the effort of the microfluidics research community
    has been spread out in exploring even 
    more technologies for microfluidic biochips, 
    %all of small size, 
    %while ignoring the potential of large system integration which 
    %they already have in hand.
    leading to a flourishing but fragmented panorama in the research on
    microfluidics. 
    
    The %state of the art in 
    status of
    the microfluidics community is similar to the early
    period of the semiconductor industry. At that time, researchers
    were exploring different materials and device structures to build smaller but
    faster transistors. Thereafter, CMOS-based technology became dominant
    in this industry, while other technologies are 
    employed only for specific applications. 
    CMOS technology obtained its dominance because of
    %, first of all, 
    its performance. 
    %However, a very important factor which assisted this dominance is that
    %the semiconductor industry and the electronic design automation community  
    %have found a way to carry out mass production of these devices and shrink the
    %feature size continuously. 
    However, the development of Electronic Design Automation (EDA) has 
    supported the large-scale integration in design and manufacturing and
    %, while  
    %with smaller devices, 
    %In the meantime,
    %Moreover, 
    %the computer community has developed a successful computing model
    %to 
    %also presented the available resources to 
    %%end users 
    %%and facilitate the development of 
    %high-level applications successfully.
    made the computing resources available to designers successfully.
     
    Observing the state of the art of microfluidic biochips, 
    researchers from computer science and 
    electrical engineering have
    started to bring their own computing models into microfluidic biochips. For
    example, the architecture of a microfluidic biochip from
    \cite{AminTA09} is shown in \figname~\ref{fig:biochip_arch}.
    In this architecture, the 
    mixer functions as
    the computing unit and intermediate results from the mixer are stored in the
    dedicated storage unit. The cells in the storage unit are built from
    normal channels. At the ports of this 
    storage unit, valves form
    multiplexers to direct fluid samples to enter into or leave from specific
    cells. This architecture emulates the classical von Neumann computer 
    %


\begin{figure}
    {
    \vskip -6pt
    \figurefontsize
    \centering
    \input{1_introduction/figures/biochip_arch.pdf_tex}
    \caption{Computing-based biochip architecture containing a mixer and a dedicated
    storage unit with eight cells \cite{AminTA09}.}
    \label{fig:biochip_arch}
    }
    \end{figure}

    
    architecture 
to build a biochemical computing system from basic components. %and devcies.
However, this simple emulation forsakes many unique characteristics of
flow-based biochips, 
% and the efficiency 
%%of this architure in 
%of executing bioassays with this architecture 
%is affected tremendously.
leading to inefficient execution of bioassays.

Similar to the semiconductor industry, design automation tool chains are also needed to 
support the 
%design and manufacturing 
development
of microfluidic biochips. In recent years,
the electronic design automation community has tried to migrate the existing
design methodologies for integrated circuits to 
microfluidic biochip design, covering the phases from 
high-level synthesis down to physical design. 
%similar to the steps shown in \figname~\ref{fig:biochip_synthesis}.
Although this top-down flow has served the semiconductor industry in the past 50 years
very successfully, fundamental changes should still be made to deal with 
specific requirements of biochips and take advantage of their unique
features.
 
%comes from the EDA community, which has
%successfully supported the rapid evolvement of semiconductor industry with
%mature design flows in the past 50 years. 

\vskip 8pt
\textit{\underline{Flow-based microfluidic biochips: the unique
characteristics}}

In microfluidic biochips, the inputs to an operation are fluid samples. 
Unlike electrical signals in integrated circuits, these fluid samples 
have a physical mass.  
%and their transportation between devices should be confined in tiny tubes, called channels. 
In executing operations of a bioassay, 
fluid samples are processed with various operations, 
such as mixing, heating and detecting in different devices. 
The results of these operations are often fluid samples of different
properties, so that inadvertent contamination between them should be avoided. 
The intermediate
results of these operations should be stored in the chip temporarily in case
they are not used immediately.
%Therefore, 
%fluid samples need to be moved between devices very often. 
%Since a device can only process fluid samples with a specific volume, 
%leading to many new samples produced in the chip.
Consequently, the physical mass and the variety of fluid samples 
become the major differences between biochips and 
integrated circuits, leading to several unique
characteristics in biochip design.

\textit{Volume Management}: In executing a bioassay, 
the volumes of fluid samples should be managed.
Assume all the devices executing the bioassay in
\figname~\ref{fig:sequencing_graph}
have a capacity $\nu$. Each of the resulting samples of $O_1$ and $O_2$ thus has a
volume $\nu$. When these two fluid samples reach the device executing
$O_7$, half of their volumes should be disposed of because the device only 
accepts a volume $\nu$.
%sample with a volume  namely, a half of the resulting samples
%from $O_1$ and $O_2$. 
%Consequently, the other half of the results should be
%discarded through some channels to a waste.
%%during the exution of the assay. 
%This volume management problem can 
%also be viewed at the assay level. For example, the operations 
%$O_1$--$O_4$ in \figname~\ref{fig:sequencing_graph}
%produce in total a volume $4\nu$, but the last operation $O_{11}$
%only produces a volume $\nu$, so that some volumes must be disposed of through 
%channels to the waste during the
%execution of the bioassay.
This volume management is not stated explicitly in the sequencing graph, but 
must be dealt with implicitly according to the volumes of intermediate fluid 
samples and the capacities of devices.    

\textit{Storage management}: In the schedule in
\figname~\ref{fig:biochip_synthesis}(a), $O_2$ completes before $O_5$ does. The
intermediate result of $O_2$ should be moved out of Mixer2 and stored
somewhere temporarily so that the mixer
can execute $O_3$. 
In the biochip shown in \figname~\ref{fig:biochip_arch}, 
this storage function is fulfilled by moving the result of $O_2$
%the intermediate fluid sample from $O_2$, 
%should be moved to a 
to the dedicated storage unit through a channel. 
%This not only
%disturbs the transportation of fluid samples between devices, but also
%increases the size of the chip due to the area taken by the storage unit and
%the valves at its ports. %as in \figname~\ref{fig:biochip_arch}.
In synthesizing biochips, if operations are not
scheduled properly, many storage requirements may appear, leading to
many transportation channels and a large storage unit. 
%In biochips, however, 
In contrast to a dedicated storage unit as shown in \figname~\ref{fig:biochip_arch},
the storage function can actually be implemented using
distributed transportation channels. %instead of a dedicated storage unit. 
In fact, a fluid sample can stay anywhere in a channel in the biochip until it is
used by the next operation. 
%Furthermore, transportaton channels in flow-based biochips can also be used as
%temporary storage to cache fluid samples. 
This is a significant difference between biochips and electronic systems, 
where intermediate data can only be stored in special memory units, either
flip-flops or RAM components. This observation can be confirmed by
the storage cells in the dedicated storage unit in
\figname~\ref{fig:biochip_arch}. These cells are built of normal channels
but with valves at each end of a channel to control the store/fetch
operations.  Instead of forming a monolithic storage unit, 
these channels and valves %in the chip 
can actually be distributed in the chip so that
they can be used for storage when required, and for transportation otherwise,
leading to better flexibility and wearing balancing. 
%Consequently, the efficiency of channels and valves can be improved
%significantly.

\textit{Washing}: Unlike electrical signals, fluid samples leave residue in
channels after they travel through them. Before such a
channel is reused by another fluid sample, it should be washed by neutral fluids
such as silicon oil. Washing contaminated channels can be very flexible
because several channel segments can be washed simultaneously if they 
form a connected graph while being isolated from the rest of the biochip
that is executing other operations.

\textit{Flow-layer and control-layer codesign}: In a flow-based biochip,
valves are controlled by air pressure through control channels, e.g., the red
channels in \figname~\ref{fig:biochip_arch}. If all the valves are controlled
independently, the routing of control channels in a complex design 
%will inevitably intersect.
becomes very complicated.
To solve this problem, control channels of some valves can be shared if
operations can still be executed correctly. This sharing requires a codesign
of the flow layer and the control layer to match the actuation patterns of 
valves.
%In addition, the routing
%of flow valves and control valves should be determined together. Otherwise,
%intersections of these channels can for further valves unintentionally.

\vskip 8pt
\textit{\underline{State-of-the-art research on design automation for
flow-based biochips using computer algorithms}}

Several methods have been proposed to synthesize 
flow-based biochips. The method in \cite{MinhassPMB12} proposes a top-down
flow to generate a biochip architecture while minimizing the execution time of
a bioassay. The flow channel routing problem considering obstacles 
is solved  with an algorithm based on rectilinear Steiner minimum tree
in \cite{LinLCLH14}.
These methods
still assume that intermediate fluid results can be stored automatically in a dedicated storage
unit as in the biochip 
inspired by electronic design 
%electronic-emulated biochip architecture 
shown 
in \figname~\ref{fig:biochip_arch}. The real storage
process and its efficiency, however, have not been investigated.

To avoid contamination, a method based on path searching 
is introduced in  \cite{HuHC16} to wash devices and channel segments. This method 
still traces path sets and block-based partial washing has not been
explored. The latter requires a co-optimization between operation scheduling
and washing activities.

Control logic synthesis is investigated in \cite{MinhassPMH13}
to reduce the number of control pins. The method 
%is improved further 
in \cite{HuDHC17} minimizes
pressure-propagation delay in the control layer to reduce the response
time of valves and synchronize their actuations.
%The control layer design of biochips has been investigated in 
Furthermore, flow layer and control layer codesign is investigated in
\cite{YaoWRCH15} to achieve valid routing results on both layers iteratively, and
length-matching in routing control channels is considered in \cite{YaoHC15} as
well. Since these methods 
do not consider operation scheduling, %during synthesis, 
the number of control
channels may still be large and consequently they might not be routed successfully.   

%Furthermore,
%fault models of manufacturing defects 
%and an ATPG-based test strategy for flow-based biochips are proposed
%in \cite{HuYHC14}.
Though the volume management problem in biochips has been explored as early as in
2008 \cite{AminTVWJ08}, and later in \cite{MitraRBCB14}
for the specific bioassay sample preparation, the optimization of volume management for general
bioassays and the interaction of this task with fluid transportation 
for normal operations have not been taken into account. 

When the unique characteristics of biochips are considered, the tasks of
synthesizing flow-based biochips are entangled with each other. Consequently,
a systematic design flow covering architectural synthesis,
control layer design, washing and volume management should be explored, which is the
major objective of this project. 

\vskip 8pt
%\textit{\underline{Preliminary work of the EDA institute at TUM}}
\textit{\underline{Preliminary work}}

Observing the great potential of microfluidic biochips and the design
automation challenges at the eve of their large-scale integration, I have 
initiated the research on biochips in the Institute for Electronic Design
Automation at TUM.
%in 2014. 
Applying the knowledge on design automation methods for IC design to 
microfluidic biochips, 
%we have 
%expanded our research onto this
%interdisciplinary topic successfully and preliminary ideas from our group have
%been verified by this preliminary work.
several preliminary ideas have been verified in our research group 
to synthesize efficient biochip architectures.

In the research community, we have pioneered the idea %of biochip architectures 
%with 
of distributed channel storage in flow-based biochips \cite{TsengLSH15}. 
%where, instead of 
%using a dedicated storage unit, intermediate
%fluid samples are cached in channel segments temporarily. 
%With this uniform
%transportation and storage model, the efficiency of biochips can be improved
%significantly even with a reduced resource usage. 
We have also proposed to improve
the reliability of biochips with a large-scale integration \cite{TBMTtcad}, 
where a fully reconfigurable valve array is used to execute operations 
and fulfill the functions of transportation and storage. 
%Additionally, we have explored the idea of flow-layer and control-layer codesign
%to simplify valve actuations as in \cite{TsengLLHS16,WZYH17}. 
Furthermore,  
we have introduced a path-based vector generation method for test of  
microfluidic biochips \cite{CBBK17}. 
%This method guarantees the detection of any two faults in a flow-based biochip. 
Fault localization and design-for-testibility
of microfluidic biochips have been addressed in \cite{Liu2018dac,aledate19}.
Moreover, optimization of control logic to improve its efficiency and the overall
portability of the microfluidic platform has been explored in \cite{Zhu2018iccad}.
%From the application view, we have investigated the %modeling of sieve valves and
%single-cell analysis bioassay executed on a hybrid microfluidic platform
%\cite{MICS17} (Best Paper Award at DATE 2017). %\cite{LiTLHS16,MICS17}. 
%(\cite{MICS17} has also been nominated for the best paper award?) 
%To explore the variety of %low-level 
%biochip
%technologies, we also investigated design automation challenges in
%biochips printed on paper \cite{WangLCKYHSLSC16}. 
%Our work on microfluidic biochips has contributed to a tutorial paper summarizing
%the state of the art and design challenges of emerging systems \cite{WilleLSD16}.

%Due to the promising perspective of biochips, we have successfully obtained
%the support of the project FLUIDA by the International Graduate School of Science and
%Engineering (IGSSE) at TUM. This project focuses on the exporation of the
%variaty in biochip components and devices and the development a system layer
%to provide uses a stable application interface. In addition, together 
%with our visiting professor Tsung-Yi Ho from National Tsing Hua university,
%we have also received the grant from the Institute for Advanced Studay (IAS) to
%support our research on cyber-physical integration of microfludic biochips.
%In August, 2015, Prof. Tsung-Yi Ho also organized a dagstguhl seminar on
%biochips \cite{}
%to connect the design automation community with the microfluidic community.
%%Furthermore, the institute of advance studay also awarded Prof. Krishnendu
%%Chakrabarty from Duke University and Prof. Tsung-Yi Ho the Hans Fischer
%%Fellow so that they will 
%The project in this proposal will form an integral part with the two projects
%above and a foreseeable sc

To bridge microfluidic biochips with their applications, 
techniques are required to map bioassays to specific
architectures. More importantly, the structures of bioassays may influence
biochip architectures because different sequencing graphs lead to different
execution, transportation and storage requirements.
%as well as their dependency. 
Therefore, efficient algorithms are also needed to optimize biochip
architecture and assay execution. 
In the past, our research group and the EDA institute 
had broad research activities in design automation for integrated circuits with
well-recognized results. The developed algorithms and tools may also
potentially benefit the research on microfluidic biochips, e.g., those for
physical design \cite{Spindler2008}, circuit test and tuning \cite{ZhangLS16} 
(Nomination for Best Paper Award at DAC 2016),
reliability \cite{Barke2015}, as well as hierarchical modeling and analysis
\cite{Li2013}. 

% \subsection{Name your subsection} % subsection headings are again smaller than section names
% % lead
% Different organized systems have different energy currencies. The machines that enable us to do science like sizzling electricity but at a controlled voltage. Earth's living beings are no different, except that they have developed another preference. They thrive on various chemicals. 

% % dextran, starch, glycogen
% Most organisms use polymers of glucose units for energy storage and differ only slightly in the way they link together monomers to sometimes gigantic macromolecules. Dextran of bacteria is made from long chains of $\alpha$-1,6-linked glucose units. 

% %: ----------------------- HELP: special characters
% % above you can see how special characters are coded; e.g. $\alpha$
% % below are the most frequently used codes:
% %$\alpha$  $\beta$  $\gamma$  $\delta$

% %$^{chars to be superscripted}$  OR $^x$ (for a single character)
% %$_{chars to be suberscripted}$  OR $_x$

% %>  $>$  greater,  <  $<$  less
% %≥  $\ge$  greater than or equal, ≤  $\ge$  lesser than or equal
% %~  $\sim$  similar to

% %$^{\circ}$C   ° as in degree C
% %±  \pm     plus/minus sign

% %$\AA$     produces  Å (Angstrom)




% % dextran, starch, glycogen continued
% Starch of plants and glycogen of animals consists of $\alpha$-1,4-glycosidic glucose polymers \cite{lastname07}. See figure \ref{largepotato} for a comparison of glucose polymer structure and chemistry. 

% Two references can be placed separated by a comma \cite{lastname07,name06}.

%: ----------------------- HELP: references
% References can be links to figures, tables, sections, or references.
% For figures, tables, and text you define the target of the link with \label{XYZ}. Then you call cross-link with the command \ref{XYZ}, as above
% Citations are bound in a very similar way with \cite{XYZ}. You store your references in a BibTex file with a programme like BibDesk.





% \figuremacro{largepotato}{A common glucose polymers}{The figure shows starch granules in potato cells, taken from \href{http://molecularexpressions.com/micro/gallery/burgersnfries/burgersnfries4.html}{Molecular Expressions}.}

%: ----------------------- HELP: adding figures with macros
% This template provides a very convenient way to add figures with minimal code.
% \figuremacro{1}{2}{3}{4} calls up a series of commands formating your image.
% 1 = name of the file without extension; PNG, JPEG is ok; GIF doesn't work
% 2 = title of the figure AND the name of the label for cross-linking
% 3 = caption text for the figure

%: ----------------------- HELP: www links
% You can also see above how, www links are placed
% \href{http://www.something.net}{link text}

% \figuremacroW{largepotato}{Title}{Caption}{0.8}
% variation of the above macro with a width setting
% \figuremacroW{1}{2}{3}{4}
% 1-3 as above
% 4 = size relative to text width which is 1; use this to reduce figures




Insulin stimulates the following processes:

\begin{itemize}
\item muscle and fat cells remove glucose from the blood,
\item cells breakdown glucose via glycolysis and the citrate cycle, storing its energy in the form of ATP,
\item liver and muscle store glucose as glycogen as a short-term energy reserve,
\item adipose tissue stores glucose as fat for long-term energy reserve, and
\item cells use glucose for protein synthesis.
\end{itemize}

%: ----------------------- HELP: lists
% This is how you generate lists in LaTeX.
% If you replace {itemize} by {enumerate} you get a numbered list.


 


%: ----------------------- HELP: tables
% Directly coding tables in latex is tiresome. See below.
% I would recommend using a converter macro that allows you to make the table in Excel and convert them into latex code which you can then paste into your doc.
% This is the link: http://www.softpedia.com/get/Office-tools/Other-Office-Tools/Excel2Latex.shtml
% It's a Excel template file containing a macro for the conversion.

\begin{table}[htp]
\centering
\begin{tabular}{ccc} % ccc means 3 columns, all centered; alternatives are l, r

{\bf Gene} & {\bf GeneID} & {\bf Length} \\ 
% & denotes the end of a cell/column, \\ changes to next table row
\hline % draws a line under the column headers

human latexin & 1234 & 14.9 kbps \\
mouse latexin & 2345 & 10.1 kbps \\
rat latexin   & 3456 & 9.6 kbps \\
% Watch out. Every line must have 3 columns = 2x &. 
% Otherwise you will get an error.

\end{tabular}
\caption[title of table]{\textbf{title of table} - Overview of latexin genes.}
% You only need to write the title twice if you don't want it to appear in bold in the list of tables.
\label{latexin_genes} % label for cross-links with \ref{latexin_genes}
\end{table}



% There you go. You already know the most important things.


% ----------------------------------------------------------------------



