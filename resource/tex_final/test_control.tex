
\subsection{Testing control layer leakage}

If there is a leakage from a control channel to another control channel,
an air pressure to close a valve also closes another valve simultaneously. 
As explained in \cite{HuYHC14}, such a fault normally happens to adjacent valves.
In the proposed method, we test potential leakage faults between the valves
surrounding a cell. This formulation can be extended easily to test leakage
problems in the control layer within a large area, though inevitably the
computational complexity will also increase.

To test whether two neighboring valves are closed at the same time, we 
need to close one valve and open the other and test whether the second can be
opened correctly by a flow path. If there is no water pressure detected at the sink, 
a control layer leakage problem is found.

In the flow path test described in Section~\ref{sec:flow_paths}, 
we have also tested whether a valve can be opened while two of other valves surrounding 
the same cell are closed. For example
the flow path in \figname~\ref{fig:cutset_model}b has tested some leakage
faults along the path. 
On this path, the valves forming the boundary of the path are closed while the
valves along the path are opened. If there is a leakage fault, no water
pressure can be detected at the sink. 

%because the opened and closed valves form leakage test pattern naturally. For
%example, in \figname, the valves pairs \{\}...\{\} surrounding the cell has
%met the pattern in which one of the valves in such a pair is open and the
%other is closed while testing whether the openned valve can be opened actually
%by the flow-path tests. 

Besides these pairs, the valves along the flow path, however, are not tested
for leakage fault, as shown in \figname~\ref{fig:cutset_model}b.
%however, do not form such a pattern in thes path tests.
To deal with these untested valve pairs, we generate further paths going through
one of the valves in an untested pair while the other valve is kept closed, 
functioning as the boundary of the new path. Similar to creating 
flow-paths in Section~\ref{sec:flow_paths}, 
%such a new path should cover as many untested pairs as possible. 
a minimal set of paths should be found to test these valve pairs.

In implementation, we first scan the flow paths generated as described in 
Section~\ref{sec:flow_paths}
to identify untested valve pairs. For each of such
valve pairs, we need to keep one valve open and the other closed. Since the
state of the opened valves are tested by flow paths, 
we revise the flow path problem in Section~\ref{sec:flow_paths} with an
additional constraint describing the states of a valve pair during the test.
Assume that the valves at $(i_1,j_1)$ and $(i_2,j_2)$ should be tested for
control channel leakage. We force one valve to open and the other to close
when constructing the new flow paths. The constraint can be written as
\begin{equation}
v^m_{i_1,j_1}+v^m_{i_2,j_2}=1. 
\end{equation}

The control layer leakage problem is solved in each subblock in the
hierarchy. Thereafter, there are still a few valve pairs at the boundary
between two subblocks that cannot be included in the control leakage test
because the flow path problem is formed for each subblock individually. These
remaining valve pairs are connected using a variant of breadth-first search to form the
test paths.
