\setcounter{page}{0} 
\pagenumbering{roman}



\clearpage
\thispagestyle{empty} 
\begin{table*}
\begin{center}
\begin{minipage}[t][21.5cm][t]{12.8cm}
%\fontsize{10}{10}\selectfont
\renewcommand{\baselinestretch}{1.2} 
\normalsize


\vspace{10pt}

Dear Editors, dear Reviewers,\\

\vspace{3pt}

Enclosed please find our manuscript entitled \textit{\papertitle}. We submit this
manuscript as a regular paper for publication in \textit{IEEE Transactions on
Computer-Aided Design of Integrated Circuits and Systems}.
%for publication in the Special Issue on Deep Physical Design Techniques for
%Next Generation Technologies, IEEE Transactions on Computer-Aided Design of
%Integrated Circuits and Systems.

\vspace{3pt}

This work builds on the method in the appended paper, which was published at the 
\textit{Design, Automation and Test in Europe} Conference in 2017 as [1].
In this manuscript, we enhance the previous work as follows: 

\vspace{3pt}

\begin{enumerate} 

  \item The model for testing Fully Programmable Valve Arrays (FPVAs) in [1] 
    has been extended to take advantage of multiple test
    ports to improve test efficiency. Accordingly, the number of test patterns
    can be decreased by up to 50\%, leading to a significant
    reduction of test cost. This extension is described in
    Section~\ref{sec:multiple_port_tree} and Section~\ref{sec:multiple_port_cut}.

  \item The extended test model is capable of generating test patterns for
    traditional flow-based microfluidic biochips, and the number of the
    resulting test patterns is much smaller than that from the other previous
    method based on ATPG, as described in Section~\ref{sec:adapt_traditional}.
   
  \item In the extension, test of control layer leakage is covered with additional
    constraints and explained in detail in Section~\ref{sec:control_layer_test}.
    In addition, long channels and obstacles are merged to facilitate 
    the generation of test patterns, as described in detail in Section~\ref{sec:walls_holes}. 

  \item Furthermore, a loop relaxation technique is introduced to improve the scalability of the
    original model based of ILP. Constraint violations are addressed by
    amending the test patterns to improve the efficiency of test
    generation. Consequently, large designs can be dealt with using this
    enhanced method, as described in Section~\ref{sec:loop_relax}.

  \item Simulation results have been extended according to the new improvements 
    to demonstrate their effectiveness and efficiency.

\end{enumerate} 

%\vspace{-10pt}
%We confirm that this manuscript has not been published or submitted elsewhere
%and all authors have approved the manuscript and agree with the submission.\\

\vspace{10pt}


We look forward to hearing from you 
and thank you very much for your time and consideration.


\vspace{25pt}

Yours sincerely,

\vspace{10pt}
Chunfeng Liu, Bing Li, Bhargab B. Bhattacharya, Krishnendu Chakrabarty,
Tsung-Yi Ho and Ulf Schlichtmann \\



\end{minipage}
\end{center}
\end{table*}

\clearpage
%\thispagestyle{empty} 
%\mbox{}
%\clearpage
\setcounter{page}{0} 
\pagenumbering{arabic}
