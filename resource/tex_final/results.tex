\section{Simulation Results}\label{sec:results}

The proposed framework was implemented in C++ and tested
using a \SI[mode=text]{3.20}{\GHz} CPU with 
\SI{8}{GB} memory.
We demonstrate
the results using FPVAs with valves in different rows and columns. The
tested arrays are shown in Table I, where the first column shows
the dimensions of the arrays and the second column shows
the number of valves. 
We used Gurobi \cite{gurobi} to solve the optimization problems in the
proposed method.

The results of testing FPVAs with a single pressure source and a single pressure sensor
are shown in Table~\ref{tb_test}.
% These arrays contain long channels for transportation and obstacle areas without
% valves. 
%The numbers of valves in the arrays are shown in the column $n_v$.  
%These numbers are calculated by multiplying
%the row number and column number and subtracting
%the number of valves not built due to channels and obstacles.
The direct approach is the method published in \cite{CBBK17}, which
does not use loop relaxation introduced in Section~\ref{sec:loop_relax} for
acceleration. In Table~\ref{tb_test}, the
numbers of test paths are shown in the column $n^d_p$. The CPU time to
generate these patterns is shown in the column $t^d_p$. The columns $n^d_c$ and
$t^d_c$ show the number of cut patterns and the CPU time to generate them.
The columns $n^d_l$ and $t_l^d$ show the number of patterns for testing leakage in the
control layer and the CPU time to generate these patterns. The
total number of test patterns is shown in the column $N^d$. 
``*'' means that no valid results are available 
%from the corresponding approach 
due to runtime.
According to these results, the direct approach suffers from the scalability problem
due to its pure ILP formulation.
%Though the flow path test patterns can still be found with the valve array of
%the size $20\times20$, the cut patterns cannot be generated.
In these results, the number of cuts is much larger than the number of test
paths. This is because a path can travel through many valves in a zigzag shape and 
only a few orthogonal paths can already cover all the valves.
However, the cuts required to cover all the valves in an
FPVA is proportional to the size of the array, leading a much larger number of
such test patterns.

\begin{figure}[t]
{\figurefontsize
\centering
\pgfplotsset{compat=1.3,
    %legend drawing style, single bar instead of the default double mini bars
    /pgfplots/ybar legend/.append style={ 
        /pgfplots/legend image code/.code={%
           \draw[##1,/tikz/.cd,yshift=-0.25em]
           (0cm,0cm) rectangle (5pt,0.8em);
        },
    }
}
\begin{tikzpicture}
\begin{axis}[
% x=1.1cm, y=0.25cm, ymax=12, line width=0.75pt,
x=1.1cm, y=1.40, ymax=55, line width=0.75pt,
xlabel={FPVAs with different size}, xlabel shift=-5pt,
ylabel={Number of test patterns}, ylabel shift=0pt, 
xtick={1,...,6},xticklabels={$5\times5$, $10\times10$, $15\times15$, $20\times20$, $25\times25$, $30\times30$},
x tick label style={rotate=330, xshift=-15pt,yshift=-8pt,anchor=west}, 
xticklabel pos=left, xtick align=outside, xtick pos=left,
ytickmin=0,ytickmax=40,
% ymin = 0,
% ymax = 1,
ytick={0,10,20,30}, yticklabels={0,10,20,30},
legend columns=4, legend style={at={(0.5,0.88)}, anchor=center, nodes={inner xsep=0pt},
draw=none, column sep=5pt},
ybar=0pt, bar width=7
]

   \addplot[line width=0.5pt, black, fill=orange!70!white] table[x=size,y=2port] 
{port_testvector_number.dat};
   \addplot[line width=0.5pt, black, fill=blue!65!orange] table[x=size,y=3port] 
{port_testvector_number.dat};
   \addplot[line width=0.5pt, black, fill=blue!40!orange] table[x=size,y=4port] 
{port_testvector_number.dat};
   \addplot[line width=0.5pt, black, fill=blue!80!orange] table[x=size,y=5port] 
{port_testvector_number.dat};
   \legend{3 Ports\hspace*{0.5pt}, 4 Ports\hspace*{0.5pt}, 5 Ports\hspace*{0.5pt}, 6 Ports\hspace*{0.5pt}}
\end{axis}
\end{tikzpicture}

%\begin{tikzpicture}
\begin{axis}[
% x=1.1cm, y=0.25cm, ymax=12, line width=0.75pt,
x=1.1cm, y=1.40, ymax=60, line width=0.75pt,
ylabel={Cut-set Test Vector Number}, ylabel shift=-6pt, 
xtick={1,...,6},xticklabels={$5\times5$, $10\times10$, $15\times15$, $20\times20$, $25\times25$, $30\times30$},
x tick label style={rotate=330, xshift=-15pt,yshift=-5pt,anchor=west}, 
xticklabel pos=left, xtick align=outside, xtick pos=left,
ytickmin=0,ytickmax=40,
% ymin = 0,
% ymax = 1,
ytick={0,5,10,15,20,25,30,35,40}, yticklabels={0,5,10,15,20,25,30,35,40},
legend columns=4, legend style={at={(0.5,0.9)}, anchor=center, nodes={inner xsep=2pt},
draw=none, column sep=1pt},
ybar=0pt, bar width=7
]

   \addplot[line width=0.5pt, black, fill=orange!70!white] table[x=size,y=2port] 
{port_cut_number.dat};
   \addplot[line width=0.5pt, black, fill=blue!65!orange] table[x=size,y=3port] 
{port_cut_number.dat};
   \addplot[line width=0.5pt, black, fill=blue!40!orange] table[x=size,y=4port] 
{port_cut_number.dat};
   \addplot[line width=0.5pt, black, fill=blue!80!orange] table[x=size,y=5port] 
{port_cut_number.dat};
   \legend{3 Ports\hspace*{0.5pt}, 4 Ports\hspace*{0.5pt}, 5 Ports\hspace*{0.5pt}, 6 Ports\hspace*{0.5pt}}
\end{axis}
\end{tikzpicture}
%%\begin{tikzpicture}
\begin{axis}[
x=1.1cm, y=0.26cm, ymax=12, line width=0.75pt,
ylabel={Exec./Valve Ratio}, ylabel shift=-6pt, 
xtick={1,...,6},xticklabels={RA100, RA70, CPA, RA30, IVD, PCR},
x tick label style={rotate=330, xshift=-15pt,yshift=-5pt,anchor=west}, 
xticklabel pos=left, xtick align=outside, xtick pos=left,
ytickmin=0,ytickmax=10, 
ytick={0,5,10}, yticklabels={0,0.5,1},
legend columns=2, legend style={at={(0.5,0.9)}, anchor=center, nodes={inner xsep=2pt},
draw=none, column sep=1pt},
ybar=0pt, bar width=7
]
   \addplot[line width=0.5pt, black, fill=orange!70!white]
table[x=cir,y=exetime] 
{exe_time_valve_cmp.dat};
   \addplot[line width=0.5pt, black, fill=blue!65!orange] table[x=cir,y=valve] 
{exe_time_valve_cmp.dat};
   \legend{Execution Time\hspace*{9pt}, Valve\hspace*{8pt}}
\end{axis}
\end{tikzpicture}

\caption{Comparison of numbers of test patterns for FPVAs with different numbers of ports.}
\label{fig:port_test}
}
\end{figure}

\begin{table*}[t] 
\centering
{\footnotesize
\renewcommand{\tabcolsep}{10.35pt}
\renewcommand{\arraystretch}{1}
\caption{Comparison between test patterns generated by the proposed method
and the ATPG-based method}
\label{tb_tradition}
\begin{tabular}{ c  c  c c c c c} \hlinewd{0.7pt}
\multicolumn{3}{c}{Proposed method} &
\multicolumn{1}{c}{} &
\multicolumn{3}{c}{ATPG-based method \cite{HuYHC14}}\\
% \multicolumn{1}{c}{} &
% \multicolumn{7}{c}{Acceleration} \\
\cline {1-3}\cline {5-7}
\multicolumn{1}{c}{} &
\multicolumn{1}{c}{Test Pattern} &
\multicolumn{1}{c}{Detected Faults} &
\multicolumn{1}{c}{} &
\multicolumn{1}{c}{} &
\multicolumn{1}{c}{Test Pattern} &
\multicolumn{1}{c}{Detected Faults}\\
% \multicolumn{1}{c}{$n^d_p$} &
% \multicolumn{1}{c}{$t^d_p(s)$} &
% \multicolumn{1}{c}{$n^d_c$} &
% \multicolumn{1}{c}{$t^d_c(s)$} &
% \multicolumn{1}{c}{$n^d_l$} &
% \multicolumn{1}{c}{$t^d_l(s)$} &
% \multicolumn{1}{c}{$N^d$} &
% % \multicolumn{1}{c}{$T^d(s)$} &
% \multicolumn{1}{c}{} &
% \multicolumn{1}{c}{$n^a_p$} &
% \multicolumn{1}{c}{$t^a_p(s)$} &
% \multicolumn{1}{c}{$n^a_c$} &
% \multicolumn{1}{c}{$t^a_c(s)$} &
% \multicolumn{1}{c}{$n^a_l$} &
% \multicolumn{1}{c}{$t^a_l(s)$} &
% \multicolumn{1}{c}{$N^a$} \\
% \multicolumn{1}{c}{$T^a(s)$} \\


\hlinewd{0.6pt}
\hline
1	&0010000011100111110000110000	&17 	&&1	&1111010101100010010110101000	&12\\
2	&1010101110111111001100001101	&5		&&2	&0111001010010111010100010011	&10\\
3	&1011100111111100111111111111	&2		&&3	&1110011110100111100111001111	&13\\
4	&1111111101111111111111111111	&1		&&4	&1110010011101110001110010011	&4\\
5	&1101111111111111111111111111	&1		&&5	&0110100110101110100011110100	&5\\
6	&1111111111011111111101111111	&1		&&6	&1111011010100111110111111000	&5\\
7	&1111111111111011111111111111	&1		&&7	&0010101010110110101101011001	&2\\
8	&1111111011101111111111101111	&24		&&8	&1010011111110101010010011100	&4\\
9	&0010000110111111000000010111	&11		&&9	&0010011010110101110100100010	&1\\
	&								&		&&10	&0011010011011111010000000011	&1\\
	&								&		&&11	&0100101011010101010001001101	&1\\
	&								&		&&12	&1010100111100011100001101011	&1\\
	&								&		&&13	&1011010011110111110000100001	&1\\
	&								&		&&14	&1110001010110111010010011001	&1\\
	&								&		&&15	&1010011011111111101011000100	&1\\
	&								&		&&16	&1011110111101111010011001000	&1\\


\hline



% 5  $\times$ 5   &39  &&1 $\times$  1&5 $\times$ 5  &&5  & 0.3  &&8  &0.2 & &  4  &  2   &&17 & 2.5 \\ 
% 10 $\times$ 10  &176 &&2 $\times$  2&5 $\times$ 5  &&4  & 4    &&18 &5   & &  4  &  10  &&26 & 19  \\
% 15 $\times$ 15  &411 &&3 $\times$  3&5 $\times$ 5  &&8  & 17   &&28 &26  & &  8  &  127 &&44 & 170 \\
% 20 $\times$ 20  &744 &&4 $\times$  4&5 $\times$ 5  &&16 & 35   &&38 &41  & & 16  & 742  &&70 & 818 \\
% 30 $\times$ 30  &1704&&6 $\times$  6&5 $\times$ 5  &&20 & 255  &&58 &171 & & 20  & 1492 &&98 & 1918\\
\hlinewd{0.7pt}
\end{tabular}
}
\end{table*}











%finding
%cuts without any acceleration method is thus infeasible due to the nature of
%ILP formulation.

The columns of loop acceleration show the results of generating test patterns using
loop relaxation. The columns $n^a_p$, $n^a_c$ and $n^a_l$ show the
numbers of test paths, cuts and leakage test patterns,
respectively.
The columns $t^a_p$, $t^a_c$ and $t^a_l$
show the CPU time needed to generate these patterns, respectively. The total number
of the test patterns is
shown in the column $N^a$. These results show that with loop acceleration 
valid test patterns can be generated for all these valve arrays, though the CPU
time increases quickly as the size of the FPVA grows.
%\textbf{For the test case $15\times15$, one extra flow path is required in the
  %loop acceleration method, due to the heuristic nature of this acceleration
  %technique. BUT IN THE RESULTS THE NUMBERS OF THE PATH TEST PATTERNS ARE THE
%SAME?} 
%. When generating cut-set test vectors,
%because the number of cuts increases with the size of the FPVA , additionally
%to Lagrangian relaxation, we generate one cut at a time to cover as many valves
%as possible. This is reason why the execution time for generating cuts is
%significantly shorter than that of the direct approach. 
For the case 30$\times$30, 
the numbers of variables and constraints 
in the ILP formulation
for generating the test patterns to detect stuck-at-0 faults, 
stuck-at-1 faults and control-layer leakage are already (17702,36910), 
(100920,1016247) and (20511,88552), respectively. 
Therefore, a long CPU time
was required to solve the corresponding optimization problems.

The results of the proposed method for multiple-port test is shown in
\figname~\ref{fig:port_test}, where 
%.  In \figname~\ref{fig:port_test}(a), 
the numbers of total test patterns for FPVAs with different numbers of test ports
are compared. 
With this comparison, it can be seen that the number of test
patterns decreases quickly as the number of test ports increases. 
This is also valid when the number of test patterns with three ports is compared
with that with two ports shown in Table~\ref{tb_test}.
Since the majority of the test patterns are cuts,
more test ports enable a better coverage of valves by cuts composed of multiple
segments,
%as shown in Fig~\ref{fig:port_test}(b).
leading to a significant reduction of test patterns.

We also 
demonstrate the effectiveness of our method in dealing with long channels
and obstacles in FPVAs. The results are shown in
\figname~\ref{fig:wall_hole_test}. In these cases, we randomly
created FPVA layouts with 5\%, 10\%, 15\% and 20\% missing valves and
obstacles. For each of these cases, one pressure source and two pressure sensors are 
assumed.
As shown in this comparison, long channels and obstacles 
have only slight influence on the number of test patterns for FPVA test.
%makes little
%difference on the number of trees and cuts needed to test all vavles, as shown
%in Fig~\ref{fig:wall_hole_test}.
However, it is worth noting that the real form of the test trees and cuts are
completely different from those without long channels and obstacles. 

\begin{figure}[t]
{\figurefontsize
\centering
\pgfplotsset{compat=1.3,
    %legend drawing style, single bar instead of the default double mini bars
    /pgfplots/ybar legend/.append style={ 
        /pgfplots/legend image code/.code={%
           \draw[##1,/tikz/.cd,yshift=-0.25em]
           (0cm,0cm) rectangle (5pt,0.8em);
        },
    }
}
\begin{tikzpicture}
\begin{axis}[
% x=1.1cm, y=0.25cm, ymax=14, line width=0.75pt,
x=1.1cm, y=9.00, ymax=9, line width=0.75pt,
xlabel={FPVAs with different size},xlabel shift=-5pt,
ylabel={Number of test trees}, ylabel shift=0pt, 
xtick={1,...,6},xticklabels={$5\times5$, $10\times10$, $15\times15$, $20\times20$, $25\times25$, $30\times30$},
x tick label style={rotate=330, xshift=-15pt,yshift=-8pt,anchor=west}, 
xticklabel pos=left, xtick align=outside, xtick pos=left,
ytickmin=0,ytickmax=6,
% ymin = 0,
% ymax = 1,
ytick={2,4,6}, yticklabels={2,4,6},
legend columns=4, legend style={at={(0.5,0.86)}, anchor=center, nodes={inner xsep=0pt},
draw=none, column sep=5pt},
ybar=0pt, bar width=7
]

   \addplot[line width=0.5pt, black, fill=orange!70!white] table[x=size,y=5] 
{hole_wall_path.dat};
   \addplot[line width=0.5pt, black, fill=blue!65!orange] table[x=size,y=10] 
{hole_wall_path.dat};
   \addplot[line width=0.5pt, black, fill=blue!40!orange] table[x=size,y=15] 
{hole_wall_path.dat};
   \addplot[line width=0.5pt, black, fill=blue!80!orange] table[x=size,y=20] 
{hole_wall_path.dat};
   \legend{5\% \hspace*{0.5pt}, 10\% \hspace*{0.5pt},15\%\hspace*{0.5pt},20\%\hspace*{0.5pt}}
\end{axis}
\end{tikzpicture}
\vskip 10pt\pgfplotsset{compat=1.3,
    %legend drawing style, single bar instead of the default double mini bars
    /pgfplots/ybar legend/.append style={ 
        /pgfplots/legend image code/.code={%
           \draw[##1,/tikz/.cd,yshift=-0.25em]
           (0cm,0cm) rectangle (5pt,0.8em);
        },
    }
}
\begin{tikzpicture}
\begin{axis}[
% x=1.1cm, y=0.25cm, ymax=14, line width=0.75pt,
x=1.1cm, y=1.40, ymin=0, ymax=49.5, line width=0.75pt,
xlabel={FPVAs with different size}, xlabel shift=-5pt,
ylabel={Number of cuts}, ylabel shift=0pt, 
xtick={1,...,6},xticklabels={$5\times5$, $10\times10$, $15\times15$, $20\times20$, $25\times25$, $30\times30$},
x tick label style={rotate=330, xshift=-15pt,yshift=-8pt,anchor=west}, 
xticklabel pos=left, xtick align=outside, xtick pos=left,
ytickmin=0,ytickmax=40,
% ymin = 0,
% ymax = 1,
ytick={10,20,30}, yticklabels={10,20,30},
legend columns=4, legend style={at={(0.5,0.86)}, anchor=center, nodes={inner xsep=0pt},
draw=none, column sep=5pt},
ybar=0pt, bar width=7
]

   \addplot[line width=0.5pt, black, fill=orange!70!white] table[x=size,y=5] 
{hole_wall_cut.dat};
   \addplot[line width=0.5pt, black, fill=blue!65!orange] table[x=size,y=10] 
{hole_wall_cut.dat};
   \addplot[line width=0.5pt, black, fill=blue!40!orange] table[x=size,y=15] 
{hole_wall_cut.dat};
   \addplot[line width=0.5pt, black, fill=blue!80!orange] table[x=size,y=20] 
{hole_wall_cut.dat};
   \legend{5\% \hspace*{0.5pt}, 10\% \hspace*{0.5pt},15\%\hspace*{0.5pt},20\%\hspace*{0.5pt}}
\end{axis}
\end{tikzpicture}

%\begin{tikzpicture}
\begin{axis}[
x=1.1cm, y=0.26cm, ymax=12, line width=0.75pt,
ylabel={Exec./Valve Ratio}, ylabel shift=-6pt, 
xtick={1,...,6},xticklabels={RA100, RA70, CPA, RA30, IVD, PCR},
x tick label style={rotate=330, xshift=-15pt,yshift=-5pt,anchor=west}, 
xticklabel pos=left, xtick align=outside, xtick pos=left,
ytickmin=0,ytickmax=10, 
ytick={0,5,10}, yticklabels={0,0.5,1},
legend columns=2, legend style={at={(0.5,0.9)}, anchor=center, nodes={inner xsep=2pt},
draw=none, column sep=1pt},
ybar=0pt, bar width=7
]
   \addplot[line width=0.5pt, black, fill=orange!70!white]
table[x=cir,y=exetime] 
{exe_time_valve_cmp.dat};
   \addplot[line width=0.5pt, black, fill=blue!65!orange] table[x=cir,y=valve] 
{exe_time_valve_cmp.dat};
   \legend{Execution Time\hspace*{9pt}, Valve\hspace*{8pt}}
\end{axis}
\end{tikzpicture}

\caption{Comparison of numbers of test patterns for FPVAs with different numbers of long channels and obstacles.}
\label{fig:wall_hole_test}
}
\end{figure}

We have applied the proposed method onto the traditional multiple-port
continuous-flow biochip shown 
in \figname~\ref{fig:test_traditional} from \cite{HuYHC14}. The port $P3$
is connected to a pressure source. The ports $P1-P2$ and $P4-P15$ are connected
to pressure sensors. 
Table~\ref{tb_tradition} shows the comparison between the test patterns generated by our
method and the test patterns generated by the ATPG method in \cite{HuYHC14} in
detecting stuck-at-0 and stuck-at-1 faults.
The test patterns define the open/closed states
of valves in the order (a--z, A, B), and the numbers of
individual faults that can be detected by the corresponding test patterns are
also shown in the corresponding rows in Table~\ref{tb_tradition}.
The proposed method covers all stuck-at-0 and stuck-at-1 faults with 9 test patterns. 
With the method of converting the same biochip architecture into a circuit 
and then applying ATPG to generate test patterns as proposed in \cite{HuYHC14}, 
16 test patterns are needed. Therefore, the method proposed in this paper has a
better efficiency in fault coverage. 
%a 78\% increase compared to our method. 


 \begin{figure}[t]
 {\figurefontsize
 \centering
 \input{Fig/test_traditional.pdf_tex}
 \caption{A traditional multiple-port continuous-flow biochip from \cite{HuYHC14}. P1-P15 are ports. a-z, A and B are valves.}
 \label{fig:test_traditional}
 }
 \end{figure}

\begin{figure*}
{\figurefontsize
\centering
\input{Fig/kill_loops_example.pdf_tex}
\caption{Constructing test trees on a \text{20$\times$20} FPVA with long channels and obstacles. (a) The original FPVA represented by a graph. (b) Long channels and obstacles are compressed into super cells. (c), (d) and (e) Three test trees with a loop in (e).  (f) The previously constructed test tree in (e) is altered to partially cover the valves on the loop. (g) The remaining valves to cover. (h) One additional test tree to cover the remaining valves.}
\label{fig:kill_loops_example}
}
\end{figure*}
 
To demonstrate the process of
 tree construction, the intermediate results of the \text{20$\times$20} FPVA
with long channels and obstacles are shown in \figname~\ref{fig:kill_loops_example}. 
The FPVA has been abstracted into a graph, where nodes represent the
cells and edges represent the valves. Randomly generated long channels and obstacles are
created in the array. 
\figname~\ref{fig:kill_loops_example}(a) shows the structure of the original
FPVA, where the red edges represent always-closed valves or 
obstacles. The blue edges represent normal valves that are expected to be
closed and opened during assay execution.
\figname~\ref{fig:kill_loops_example}(b) is the FPVA with its long channels and
obstacles merged 
%into regular cells with the method introduced in
as described in Section~\ref{sec:walls_holes}. \figname~\ref{fig:kill_loops_example}(c), \figname~\ref{fig:kill_loops_example}(d) and \figname~\ref{fig:kill_loops_example}(e) are
the test trees constructed using the ILP model 
%\text{(\ref{eq:ilp_tree_1})-(\ref{eq:ilp_tree_2})} 
with loop relaxation. There is a disjoint loop in \figname~\ref{fig:kill_loops_example}(e) and it is altered  
in \figname~\ref{fig:kill_loops_example}(f).
\figname~\ref{fig:kill_loops_example}(g) shows the remaining valves that need
to be covered by additional trees
after the loop is altered. In \figname~\ref{fig:kill_loops_example}(h), one
additional test tree is constructed to cover these valves.

To verify whether the test paths, cuts and the patterns for testing control
layer leakage can detect faults reliably, for each FPVA in
Table~\ref{tb_test} we generated one, two, three, four and five faults
separately and tested whether the test patterns can detect these faults. We
repeated this process for \num[mode=text]{10000} times.  In these test cases,
the test patterns detected all the faults, 
%meaning that these test patterns are very effective in detecting faults.
%compared with the in-theory guaranteed two-fault detection. 
demonstrating a very high reliability in fault detection.

