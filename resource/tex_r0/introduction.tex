\section{Introduction}

Microfluidic biochips have revolutionized the traditional slow and error-prone
biochemical experiment flow by manipulating fluids at nanoliter level. With
this miniaturization, bioassays can be scaled down and executed by moving tiny
fluid samples between exact locations. Operations for manipulating fluid
samples, e.g., split,  move, mix, and detect, are available in these chips to
implement complex bioassays. The execution of bioassays  is coordinated by
microcontrollers, so that  a high efficiency and precision can be achieved
\cite{JMSQ07,JEMP08,KrCh10}.   On biochips, genomic bioassay protocols, such as
nucleic-acid isolation, DNA purification, and DNA sequencing, have been
demonstrated successfully. In addition,  these technologies have attracted a
lot of commercial attention, such as from Illumina \cite{illumina} and Agilent
\cite{agilent}.

%Microfluidic biochip technologies have many advantages due to its
%miniaturization.  Since all the atomic operations are performed at the
%nanoliter level, only tiny amount of reagents are required, thus reducing the
%cost of the whole experiment.  
%In addition, fluid samples
%and reagents of tiny volumes can be moved very quickly on biochips and react within a very
%short time, leading to a fast response time for such chips. 
%Furthermore, this miniaturization enables the
%support of a large number of reactions in an experiment with the same volumes of
%samples in the traditional test flow, so that complex bioassays such as
%exhaustive disease diagnosis may become possible. 
%Moreover,
%with this miniaturization, large system integration may finally become feasible
%to enable complex bioassays 

%The state-of-the-art microfluidic technologies can be categorized into two
%groups.
%In the first group, fluid droplets are located on a patterned array of electrodes and
%moved by the force of electrowetting. 
%This technique has the advantage of dynamic reconfiguration
%but it is limited when dealing
%with some applications due to the applied high voltage.  The second group of
In flow-based microfluidic biochips, 
%which we will focus on in this paper, use
microvalves are used to control the movement of fluids.  
The structure of a microvalve
is shown in \figname~\ref{fig:valve_mixer_storage}(a).  In this structure, a
flow channel is constructed on a substrate for the transportation of fluids.
Above the flow channel, a control channel is constructed and
connected to an air pressure. Since both channels are built from
elastic materials, air pressure applied in the control channel squeezes the
flow channel tightly, so that the movement of the fluid segment is blocked.
If the pressure in the control channel is released, the fluid
segment can resume its movement to the target device. Consequently, a valve is
formed at the intersection of the two channels to control fluid movement.
%Compared with electrowetting-driven biochips, valve-based biochips
%can execute nearly all experiments so that they have
%become the focus of the research community in recent years. 

Valves can also be used to build complex devices. For example, the
structure of a mixer is shown in \figname~\ref{fig:valve_mixer_storage}(b).  If
the three valves at the top of the mixer are actuated alternately by applying
and releasing air pressure in their control channels in a given pattern similar
to peristalsis, a
circular flow around the device can be formed to mix different fluids.
%These valves are thus called peristalsis valves. 
Furthermore, 
storage units can be constructed from normal flow channels with multiplexer-like
controlling valves at each port. These units can be used to store intermediate
results of operations temporarily.
\figname~\ref{fig:valve_mixer_storage}(c) shows a schematic of a mixer
connected to a storage unit comprising eight cells \cite{AminTA09}. 
%Consequently, only one fluid sample can enter or leave the 
%storage unit at a given moment.

%Accompanying the rapid advance of flow-based biochips, 
In view of the advantages of flow-based biochips, design automation methods for
them has gained much attention recently. For architectural synthesis, 
the method in \cite{MinhassPMB12} proposes a top-down
flow to generate efficient biochip architectures, while 
the methods in \cite{TsengLSH15,Liu2017} explore the concept of
distributed channel storage to improve execution efficiency.
The flow channel routing problem considering obstacles 
is solved using an algorithm based on rectilinear Steiner minimum tree
in \cite{LinLCLH14}.
To avoid contamination in executing operations, 
path searching is used in \cite{HuHC16} to generate washing solutions for 
devices and channel segments. 
Control logic synthesis is investigated in \cite{MinhassPMH13,WZYH17,Zhu2018iccad}
to reduce the complexity of control layer for switching valves.
In addition, pressure-propagation delay in the control layer is 
minimized in  \cite{HuDHC17} to reduce the response
time of valves and synchronize their actuations.
%The control layer design of biochips has been investigated in 
Furthermore, flow layer and control layer codesign is investigated in
\cite{YaoWRCH15} to achieve valid routing results on both layers, and
length-matching in routing control channels is investigated in \cite{YaoHC15}.
Moreover, fault models of manufacturing defects 
and an ATPG-based test strategy for flow-based biochips are proposed
in \cite{HuYHC14,HuHC14}. 
Fault localization and design-for-testibility
for flow-based biochips have been addressed in \cite{Liu2018dac,aledate19}.

The chip structure demonstrated in \figname~\ref{fig:valve_mixer_storage}(c)
is designed by researchers manually
%in the microfluidic field with 
using preliminary tools such as AutoCAD to draw the channels at different
layers. 
%After the schematic is finished, a chip is manufactured by
%etching the substrate on which fluid and control channels are formed. This
%design flow, however, only works with chips of small scale containing only 
%a limited number of channels and devices. 
%%To take advantage of 
With the advances of manufacturing technologies, many thousands of valves can
already be integrated into a single chip.
%as small as a coin. 
%the locations of
%devices and their channel connections can only be determined by computer
%algorithms for an efficient design.  Another limitation of the traditional
%biochip structure in \figname~\ref{fig:valve_mixer_storage}(c) comes from the
%irregularities of devices. 
It is thus challenging to design a whole biochip with the irregular structure
shown in \figname~\ref{fig:valve_mixer_storage}(c).
%Similar to 
%the case in 
Consequently, 
%of components, however, is not considered in 
%However, the irregular architecture of 
%traditional biochip design.
%the round mixers
%and the irregular transportation channels 
%in the traditional architecture 
%such as shown in \figname~\ref{fig:valve_mixer_storage}(c) 
%still hinder the further improvement of integration for large-scale
%biochemical analyses.
%Noticing the requirements of system integration, researchers 
%in the microfluidic society 
%have been exploring different chip architectures. 
%For example, 
%has enabled the development of 
fully programmable valve arrays (FPVAs)
%The concept of a regular structure has been 
%have been demonstrated in 
%attempted in the early work
%Quake et. al 
have emerged with a regular structure to make the large number of valves and
channels available to bioassays \cite{JMSQ07,matrix11}.
%and the first prototype toward large-scale integration has
%been demonstrated in the work Maerkl et. al 
\figname~\ref{fig:archi}(a) shows a part of the large FPVA demonstrated in
\cite{matrix11}. A drawing of partial enlargement of four valves controlling
the four directions of the fluid segment  in an enclosed fluid cell is shown in
\figname~\ref{fig:archi}(b). In this architecture, valves (solid blocks) are
arranged regularly along horizontal and vertical flow channels (light color).
These valves are controlled by air pressure through the control
channels (narrow channels). By opening two valves and closing the other two,
the fluid segment inside a cell can be moved to the intended direction.
%for transportation. 
Consequently, flexible flow paths can be formed by opening
and closing a set of valves, as shown in \figname~\ref{fig:archi}(c). 
%the semiconductor industry,
%such a large-scale integration usually involves in a regular component structure, at least at
%physical level. 
%This regularity 

\begin{figure}[t]
{\figurefontsize
\centering
\input{Fig/valve_mixer_storage_own_mixer.pdf_tex}
\caption{Components and structure of flow-based biochips. (a)
Microvalve structure. (b) Mixer. (c) Biochip with eight
storage cells \cite{AminTA09}.}
\label{fig:valve_mixer_storage}
}
\end{figure}

\begin{figure}[t]
{\figurefontsize
\centering
\input{Fig/archi.pdf_tex}
\caption{Fully programmable valve array (FPVA) \cite{matrix11}.
(a) Architecture. (b) Valve and cell. (c) Flow path construction. }
\label{fig:archi}
}
\end{figure}

\begin{figure}[t]
{\figurefontsize
\centering
\input{Fig/dynamic_devices.pdf_tex}
\caption{FPVA with dynamic devices. (a) A 4$\times$2 dynamic mixer.
(b) A 2$\times$4 dynamic mixer. (c) Dynamic mixers of different orientations
sharing the same area. (d) FPVA with a long channel and always-closed
valves (obstacles). The x-axis and y-axis show the coordinates of the valves and
cells.}
\label{fig:dynamic_devices}
}
\end{figure}


Besides flow paths, devices such as mixers can also be constructed on 
an FPVA, taking advantage of its flexibility and reconfigurability.
For example, a 4$\times$2 mixer and a 2$\times$4 mixer can be constructed as
shown in 
\figname~\ref{fig:dynamic_devices}(a) and \figname~\ref{fig:dynamic_devices}(b),
respectively. In such a dynamic mixer, the eight valves along the enclosed channel 
function as peristalsis valves, which switch in a given pattern 
to drive the fluid segment inside the channel. Compared with the
traditional mixer shown in \figname~\ref{fig:valve_mixer_storage}(b), these dynamic
mixers have different shapes and more peristalsis valves, eight in each case, to
form a 
%strong 
circular flow for mixing.
Furthermore, the two mixers in 
\figname~\ref{fig:dynamic_devices}(a) and \ref{fig:dynamic_devices}(b)
can share the same chip area as shown in \figname~\ref{fig:dynamic_devices}(c)
if the two mixers are not used at the same time, providing more
flexibility to schedule operations on such a chip.
%On valve arrays, further special devices such as heaters and detectors can 
%also be built at certain locations. In addition, fluid samples 
%can be transported between devices by forming a temporary flow path
%between two devices. After the transportation is finished, the flow path can
%be released and the valves can be reused for other functions.
The videos shown in \cite{PMD_mixing, PMD_transportation} demonstrate
real cases of dynamic device mapping and fluid transportation on an FPVA.

FPVAs have a significant advantage for large-scale integration
due to their regular structure.
%compared with the traditional nonregular architecture with dedicated round mixers as shown in 
%\figname~\ref{fig:valve_mixer_storage}(c). 
%A counterpart of this regular structure is the DRAM %(random-access memory)
%array in the semiconductor industry, where the manufacturing technology of DRAM
%chips is normally a generation more advanced than that of other chips, e.g.,
%microprocessors. Furthermore, 
The ability of dynamic reconfigurability gives
them the convenience to execute nearly any bioassays, provided specific devices
such as filters and heaters are also built in the chips. 
%%This a characteristic very similar to the one of FPGA (field programmable gate
%%array) in semiconductor industry, and it 
%This allows chip vendors to focus on improving the integration scale without
%worrying about the applications. On the other hand, the increasing integration
%of such reconfigurable chips provide customers the ability to execute very
%complex experiments automatically, 
%This is a great advantage specially for small
%%%%tributary local
%hospitals or research units which usually cannot afford expensive equipment.
%In addition, this flexibility endows them with a great potential 
%of fault tolerance, since defected valves and cells can be circumvented easily
%by scheduling operations onto other regions on a chip.
%
%
%In view of the potential of FPVAs,
%programmable valve arrays, 
%For valve arrays specifically, 
%For FPVAs specifically,
%design solutions have also started to appear for these chips. 
To facilitate the application of FPVAs, 
the method in \cite{TsengLHS15,TBMTtcad} explores
dynamic mapping of operations to reduce the
maximum number of switching activities of valves, so that valve wearing can be balanced
across all the valves in a chip.
%Test paths and cuts for detecting defects in valve
%arrays are explored in \cite{CBBK17}.  In addition, 
In \cite{pump}, flow routing considering
pressure-routes in establishing transportation paths on an FPVA is investigated
to achieve a better assay completion time. In addition, a
close-to-optimal physical design solution can also be achieved by adapting the
formulation based on satisfiability in \cite{grimmer2017close}.

%\subsection{Contributions}

%For testing valve arrays, the method in \cite{CBBK17} 

%As mentioned above, 
In the manufacturing process of FPVAs, defects may appear 
at valves and in flow and control channels.
%
%in some biochips after manufacturing.
Therefore, 
an efficient design automation method is also required to generate test patterns
%These test patterns are applied to the chips after manufacturing 
for identifying chips with defects. To reduce test cost, the number of these test
patterns should be as small as possible.
In this paper, we address this test generation problem for FPVAs 
with the following contributions:

\begin{itemize}

\item The first systematic formulation of test strategy with test paths and cuts 
  is proposed to enable efficient test of FPVAs after manufacturing.

\item The generation of test patterns
  %Cut generation 
  considers fault masking 
  %between test patterns 
  to cover multiple faults in the flow layer and the control layer.

\item Leakage in control channels is covered by additional test
  paths that are orthogonal to the test paths for stuck-at-0 faults.
  
\item Multiple ports of FPVAs are taken advantage of to improve test
  efficiency. 
  
\item The proposed test framework is completely compatible with existing test
  framework for traditional flow-based biochips,
  %such as shown in \figname~\ref{fig:valve_mixer_storage}(c), 
  so that no additional cost of the test platform is incurred. 
  
\item  When adapted to test traditional biochips, the number of test patterns
  is significantly smaller than that from previously methods based on ATPG, leading
  to a higher test efficiency.

\end{itemize}


The rest of this paper is organized as follows. In
Section~\ref{sec:formulation} we review the state of the art of
testing traditional flow-based biochips and formulate the test problem for FPVAs.
In Section~\ref{sec:test_strategy} we present the general test strategy 
for FPVAs. 
%proposed framework and its implementation.
In Section~\ref{sec:path_cut}, the models and their implementation for generating test paths and cuts
are explained in detail.
In Section~\ref{sec:multi_port}, test models are extended to cover biochips
with multiple ports and to adapt the proposed method to test traditional flow-based biochips.
Simulation results are shown in Section~\ref{sec:results} and
conclusions are drawn in Section~\ref{sec:conclusion}.

