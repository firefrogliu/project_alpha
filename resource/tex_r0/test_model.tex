
\section{Generating test-path and test-cut patterns} 
\label{sec:path_cut}

The objective of generating path and cut test patterns is
to minimize the number of test patterns to reduce test cost. The path patterns
together should cover all valves to detect stuck-at-0 faults, and so  
are the cut patterns to detect stuck-at-1 faults. Furthermore, leakage in
control layers causes two valves to switch simultaneously, so that such faults
should also be covered by the test patterns. The test patterns described in
this section assume that there is a single pressure source and a single
pressure sensor for fault test. Test patterns with multiple sensors will be discussed in
Section~\ref{sec:multi_port}.


\subsection{Constructing path test patterns} 
\label{sec:flow_paths}
%\label{sec:flow_path_cons}

In an FPVA, a test path can pass through a valve from any of the two
directions.  
%Therefore, the task to construct test paths is similar to finding
%a minimum set of paths covering all nodes in an undirected graph.
%, an NP-hard problem. 
The test paths together should cover each valve in the chip at least once.
In the proposed method, we describe this path generation using an Integer Linear
Programming (ILP) model.  The scalability of this model is further improved
using a heuristic loop removal technique. 

\subsubsection{Path pattern formulation} \label{sec:flow_path_cons}

In an FPVA, a fluid cell is defined as the channel area surrounded by four
valves. A test path can enter such a   
%separated into small cells as shown in
%\ref{fig:archi}(b). Such a cell is surrounded by four valves, and a flow
%path can enter this 
cell and leave it from any of the four valves surrounding the cell.
%.as shown in \figname~\ref{fig:flow_path_model}(a). 
%Air pressure through a cell must pass through
%two of the valves surrounding this cell. 
Consequently,  
%Since the air pressure can 
%reach
%%enter and leave
%the cell in any direction, in total 
there are 12 possible directions for a path passing through a cell, 
as illustrated by the dashed lines in \figname~\ref{fig:flow_path_model}(a).
%Instead of modeling these combinations directly, 
%we model how the path passes through the surrounding valves. 

In describing the path model, a valve
and a fluid cell in an FPVA are denoted as $\mathtt{V}_{i,}$ and
$\mathtt{C}_{i,j}$, respectively, where $(i,j)$ is the coordinate as shown in
\figname~\ref{fig:dynamic_devices}(d).
Assume all valves can be covered by 
no more than $n_p$ test paths, 
%in the set of test patterns,
where $n_p$ is a given constant. 
For the cell at the location $(i,j)$, we assign
a 0-1 variable $c^m_{i,j}$ to represent whether the $m$th path travels through
the cell. If the $m$th path travels through the cell,
$c^m_{i,j}=1$; otherwise $c^m_{i,j}=0$.
For the valves at the left, right, upper and lower sides of the cell, 
%at the location $(i,j)$, 
we assign 0-1 variables 
$v_{i, j-1}^m$, $v_{i, j+1}^m$, $v_{i+1, j}^m$ and, $v_{i-1, j}^m$, respectively. If 
the $m$th path travels through a valve, the corresponding variable is set to 1;
otherwise, it is set to 0.  

If the $m$th path travels through
the cell $\mathtt{C}_{i,j}$ at the location $(i,j)$, this path should travel through exactly 
two valves that surround the cell.
%otherwise, no valve surrounding the cell should be passed. 
Consequently, the relation between the cell and the valves surrounding it can be
established as
\begin{align}
\label{eq:valve_cell}
v_{i, j-1}^m + v_{i, j+1}^m +& v_{i+1, j}^m + v_{i-1, j}^m=2c^m_{i,j}, \\  
%&\forall\ i=2,\dots, n_r-1,\  j=2,\dots, n_c-1,\ m=1, 2,\dots, n_p
&\forall\ \mathtt{C}_{i,j}\in \mathbf{C}, \ m=1, 2,\dots, n_p\nonumber
\end{align}
%where $n_r$ and $n_c$ are the numbers of rows and columns 
where $\mathbf{C}$ is the set of all the cells in the FPVA.
$(i,j)$ is the coordinate of the cell $\mathtt{C}_{i,j}$. 
%as shown in \figname~\ref{fig:dynamic_devices}(d). 
$n_p$ is the maximum number of the
test paths.

To initiate a test path, we set the variable 
$v^m_{i,j}$ of the valve and the variable  
$c^m_{i,j}$ 
of the cell that are connected to the pressure source always 
to 1. Similarly, the variables for the valve and the cell connected to the
pressure sensor are initialized. 
The constraint (\ref{eq:valve_cell}) forces two valves neighboring a cell to
appear on a test path, and if a valve appears on a path, two cells
neighboring it must also appear on the path.
This chaining effect of the variables propagates further until the pressure
sensor is reached, thus defining a whole test path from the pressure
source to the sensor.
 \figname~\ref{fig:flow_path_model}(b) shows a partial example of this
chaining propagation defined by (\ref{eq:valve_cell}).
%Constraint (\ref{eq:valve_cell}) then forces one of the surrounding valves to
%to appear on the path, which further force the next fluid cell to appear on the
%path. 

To guarantee that a valve is covered at least once by the test paths, 
one of the constraint variables $v^m_{i,j}$ for the valve 
$\mathtt{V}_{i,j}$ 
%indexed by
%$(i,j)$ 
out of the $n_p$ paths 
%on the $m$ paths 
must be 1, leading to
\begin{equation}
\label{eq:valve_cov}
\sum_{m=1}^{n_p}v^m_{i,j}\ge 1, \ \forall\ \mathtt{V}_{i,j}\in \mathbf{V}
\end{equation}
where $\mathbf{V}$ is the set of all the valves in the FPVA and $(i,j)$ is the
coordinate of the valve $\mathtt{V}_{i,j}$.


 
%In establishing the relation between valves and the cells, there are several
%special cases. First, the cell connecting the pressure source or the pressure
%sensor must be on all the paths, so that all the constraint variables
%$c^m_{i,j}$ for these cells should be set to 1.  Accordingly, the 0-1
%variables for the virtual valves through which the pressure source and the
%pressure sensor are connected should also be 1.  
%Another special case is that

%On the FPVA, at some locations long channels instead of valves are built 
%and at some other locations obstacles without valves may exist, as illustrated
%in \figname~\ref{fig:dynamic_devices}(d). These special cases may appear due to
%design specifications or as the result of other test procedures after
%manufacturing. These cases, however, can still be incorporated into the test
%pattern generation in the proposed method, by assuming virtual valves appearing
%at these locations but their states are always open or always closed.
%Therefore, the basic relation between valves and cells (\ref{eq:valve_cell}) and
%the coverage condition (\ref{eq:valve_cov}) still hold.

\subsubsection{Excluding disjoint loops from test paths}\label{sec:disjoint_loop}

With the constraints (\ref{eq:valve_cell}) and (\ref{eq:valve_cov}),
disjoint loops may appear on a test path. 
For example, these constraints do not prevent the disjoint loop at the lower right side of the
FPVA in \figname~\ref{fig:flow_path_model}(c) from happening. All the valves and cells on
this loop meet the constraints (\ref{eq:valve_cell}) and (\ref{eq:valve_cov}),
but this loop leads to a false valve coverage in test,
because test pressure from the source cannot reach 
any valve on this loop to verify whether it can be opened.

The disjoint loop can be removed by forcing a flow from the pressure source
to any segment of the path. Assume that the pressure source needs to provide one
unit of pressure volume to fill a fluid cell, the total pressure volume stored
on a path should be equal to the number of fluid cells on the path. If the path
is not a single path but contains a disjoint loop, the number of fluid cells on
the path should be larger than the pressure volumes from the source, since
no pressure volume can reach the cells on the loop. 

When applying a test path onto an FPVA, the pressure source provides pressure
volumes that flow through the path.
To represent the pressure volume passing through a valve at the
location $(i,j)$,  we define an integer variable $f^m_{i,j}$.  This variable is
positive when viewed from a cell which the pressure flow enters; it is negative
when viewed from a cell which the pressure flow leaves. In addition, the pressure
propagation can pass through a valve only if the valve is on the test path, under the
condition $v^m_{i,j}=1$. Otherwise, $f^m_{i,j}$ must be set to 0. This
condition constrains $f^m_{i,j}$ as
\begin{equation}
  \label{eq:flow_var}
  f^m_{i,j}\le v^m_{i,j}\cdot\mathcal{M} \quad \text{and}\quad f^m_{i,j}\ge
  -v^m_{i,j}\cdot\mathcal{M} 
\end{equation}
where $\mathcal{M}$ is a large positive constant %for any flow value
\cite{chen2011applied}.

Since a fluid cell is surrounded by four valves in general,
the pressure volume stored in a cell at the location $(i,j)$ 
when the $m$th test path is applied is equal to the sum of the
volumes flowing through the four surrounding valves. 
This relation can be written as
\begin{equation}
\label{eq:flow_sum}
f^m_{i,j-1}+ f^m_{i,j+1}+ f^m_{i+1,j}+ f^m_{i-1,j} = c^m_{i,j}
\end{equation} 
where the valves on the left of, on the right of, above and below the cell
$\mathtt{C}_{i,j}$ at the location
$(i,j)$ are indexed by $(i,j-1)$, $(i,j+1)$, $(i+1,j)$ and $(i-1,j)$,
respectively. 
%The definition can also describe the condition for fluid cells
%with fewer surrounding valves such as 

%To guarantee that the water from the source can reach all cells
%on the path, the saved water volume on the path should be equal to the number
%of cells on the paths, written as
%\begin{equation} 
%\label{eq:flow_path_volume} 
%\sum_{(i,j)\in I} (f^m_{i,j-1}+ f^m_{i,j+1}+ f^m_{i+1,j}+ f^m_{i-1,j})=
%\sum_{(i,j)\in I} c^m_{i,j}
%\end{equation} 
%where $I$ is the index set $\{(i_1,j_1)\dots (i_n,j_n)\}$ for the all cells 
%in the array. In some cells are not selected to be a part of the $m$th path,
%the corresponding constraint variables $c^m_{i,j}$ are set to 0, which
%further constrain all flows to 0 by (\ref{eq:flow_var}).

%For a cell that is not selected to be a part of the path, its constraint 
%variable $(c^m_{i,j})$ is equal to 0, so that 
%constrianed by
%(\ref{eq:valve_cell}), $c_k$ is zero and all the flow variables related to
%this cell can take any value to meet (\ref{eq:valve_cell}). 
%When the $k$th
%cell is on the $m$th path, $c_k=1$. 
%(\ref{eq:valve_cell}) can thus 
%shown in \figname~\ref{fig:flow_path_model}. 

Constraint (\ref{eq:flow_sum}) is capable of preventing disjoint loops from appearing
effectively.
%Consider that we have 
Assume there is a disjoint loop on the $m$th path and the cells
on the disjoint loop are $\mathtt{V}_{i_1,j_1}, \mathtt{V}_{i_2,j_2},\dots 
\mathtt{V}_{i_l,j_l}$, where the valves 
$\mathtt{V}_{i_1,j_1}$ and $\mathtt{V}_{i_l,j_l}$ are neighbors to form a loop.
For each cell on the loop, a constraint in the form of
(\ref{eq:flow_sum}) is created. Adding the left and right sides of these constraints
together, we have
\begin{equation} 
\label{eq:flow_sum_loop} 
\sum_{(i,j)\in I_l} (f^m_{i,j-1}+ f^m_{i,j+1}+ f^m_{i+1,j}+ f^m_{i-1,j}) = 
\sum_{(i,j)\in I_l} c^m_{i,j}
\end{equation} 
where $I_l$ is the index set  $\{(i_1,j_1), (i_2,j_2), \dots (i_l,j_l)\}$ 
for the cells on the loop.
%
%For a cell on the disjoint loop, only two of the flow variables cannot be
%zero as specified by the constraints (\ref{eq:valve_cell}) and (\ref{eq:flow_var}). 
%Assume for the cell at $(i_1,j_1)$ and the cell next to it on the right is
%also on the path and its location is $(i_2,j_2)$. Also assume that
%the valve on right side of the cell at $(i_1,j_1)$ 
%has a flow value equal to $f^m_{i_1,j_1+1}$. This valve is the valve on the
%left side of the cell at $(i_2,j_2)$. Viewed from the second cell, the flow value 
%$f^m_{i_2,j_2-1}$ is equal to $-f^m_{i_1,j_1+1}$ because this is actually the
%same flow value viewed from two neighboring cells.
%reversed directions.
%In the sum on the left side of (\ref{eq:flow_sum_loop}), a flow on a valve along
%the path appears twice with reversed valves. Therefore, 
On a disjoint loop, the sum on the left
side of (\ref{eq:flow_sum_loop}) is always equal to 0, because no pressure
flow enters the loop. This contradicts 
the fact that $\sum_{(i,j)\in I_l} c^m_{i,j}$ is always larger than 0 if the
cells are on the test path.
%the constraint (\ref{eq:flow_path_volume}) because  
%$\sum_{(i,j)\in I_l} c^m_{i,j}$ is a part of $\sum_{(i,j)\in I} c^m_{i,j}$ and
%must be larger than zero.
%Assume that the $f_{i_1, l,m}> 0$, so the the flow enters the $i_1$the cell
%from the left side. For the $i_n$th cell, the flow variable for the vavle on
%its right is eqaul to $0-f_{i_1, l,m}$, because it is the same flow viewed
%from another direction. This relation is valid for all valves on the disjoint
%loop, so that the sum on the left of (\ref{eq:flow_sum_looop}), so that no
%varialbes on the right of (\ref{eq:flow_sum_looop}) can be 1 because all
%$c_k$s are 0-1 variables. This contradicts 
%with the fact that the disjoint loop is a part of the test path so that the
%right side of (\ref{eq:flow_sum_loop}) is larger than 0. 
%because at least one $c^m_{i,j}$ should be 1 to allow the flow path to pass at
%least one of the valves surrounding this cell. Consequently, disjoint loops can
%be excluded by (\ref{eq:flow_sum_loop}) in the
%optimization problem. 
Therefore, constraint (\ref{eq:flow_sum}) for each cell guarantees that all
test paths are simple paths without disjoint loops to avoid false coverage
during test. The concept of 
%flow variables and the sum
%of the flow around the loop are 
this model is illustrated in \figname~\ref{fig:flow_path_model}(d).

\begin{figure*}[t]
{
\figurefontsize
  \begin{minipage}[b]{0.70\textwidth}
\centering
\input{Fig/minimum_loops.pdf_tex}
\caption{Eliminating a disjoint loop. (a) A test path containing a disjoint loop.
(b) Altered test path partially covering valves on the loop. (d) An additional
test path created to cover the rest valves on the loop.}
\label{fig:minimum_loops}
\end{minipage}
\hspace{10pt}
  \begin{minipage}[b]{0.28\textwidth}
%    \centering
\hskip 30pt\input{Fig/cut_var.pdf_tex}
\vspace{10pt}
    \caption{Constraint variables for cut-set modeling.}
    \label{fig:cut_var}
  \end{minipage}
}
\end{figure*}

\subsubsection{Finding the minimum set of test paths} 

The path constraints defined above assume that the number $n_p$ of test paths to
cover all valves are known. 
%denoted as $n_p$. so that the path constraints described can be created for each path.
In our formulation, we first set $n_p$ to a constant and then
find a set of paths whose number is no larger than $n_p$ to cover all
valves. For each path, we
assign a 0-1 variable $p_m,\, 1 \le p_m \le n_p$ to indicate whether this path is used. Because any
valve on the $p_m$th path marks the path to be used, $p_m$ can be
constrained as
\begin{equation}   
\label{eq:valve_on_path}
p_m\cdot\mathcal{M} \ge \sum_{(i,j)\in I}v^m_{i,j}
\end{equation}   
where $I$ is the index set of all valves.
$\mathcal{M}$ is a positive constant larger than the number of valves in
the array. 
If a valve is on the $m$th path, the right side of
(\ref{eq:valve_on_path}) is larger than 0, so that $p_m$ must be 1 to
meet the constraint. 

With the constraints above, the ILP problem to find a minimum set of test paths 
that cover all valves can be formulated as follows
\begin{align} 
  \text{minimize} &\quad \sum_{m=1}^{n_p}p_m
\label{eq:ilp_1}\\
\text{subject to} & \quad (\ref{eq:valve_cell})-(\ref{eq:flow_sum})
\  \text{and}\  (\ref{eq:valve_on_path}).
\label{eq:ilp_2}
\end{align} 

Since the number of paths $n_p$ is specified as a constant in the formulation, 
it is possible that
the ILP problem above has no solution, meaning that not all the valves can be
covered by $n_p$ test paths. If this happens, we increase $n_p$ and solve the
optimization problem again. 
%This process is repeated until we find the first minimum set of feasible paths. 

\subsubsection{Relaxing constraints for excluding disjoint loops and heuristic
path reconstruction}\label{sec:loop_relax}

To exclude disjoint loops from test paths, constraint (\ref{eq:flow_sum}) is
created for each cell and included in the optimization problem
(\ref{eq:ilp_1})--(\ref{eq:ilp_2}). These strict constraints,
however, 
%burden the optimization in finding the minimum set of test paths much and 
lead to huge computation overhead. To address this issue, we introduce a
technique to relax these strict constraints to accelerate the 
computing process. 
%in excluding disjoint loops.

The basic idea of this technique is similar to Lagrangian relaxation, in which
a part of the constraints are moved into the objective and 
violations of these constraints 
%after solving the relaxed problem 
are punished by weights. 
The solution of a relaxed problem
approximates the solution of the original problem, though violations of the
original constraints may not be avoided completely. 
%These violations may be reduced by
%applying Lagrangian relaxation iteratively, 
In the proposed framework, we deal with these constraint violations
by amending the solution heuristically.
%at the cost of the quality of the solution of the optimization.
%since the quality deterioration of the solution only leads to a few more test paths.

In the disjoint loop constraint (\ref{eq:flow_sum}), the sum of the pressure
volume going through the loop must be equal to the number of the cells. 
To relax this
constraint, we allow this sum to be smaller than the number of cells, as
\begin{equation}
\label{eq:disjoint_loop_violation}
f^m_{i,j-1}+ f^m_{i,j+1}+ f^m_{i+1,j}+ f^m_{i-1,j} \le c^m_{i,j}.
\end{equation}
To prevent disjoint loops from happening, we 
%still try to construct test paths meeting (\ref{eq:flow_sum_loop}) by minimizing the 
minimize the difference between the left side and the right side of
(\ref{eq:disjoint_loop_violation}). This difference is written as
%In our case, we allow the constraints (\ref{eq:flow_sum}) to be violated. Here
%we use a 0-1 variable $r^m_{i,j}$ to denote the total flow running through the
%cell $c_{i,j}$ in the $m$th flow path. The constraints (\ref{eq:flow_sum}) are
%altered to
%\begin{align}
%\label{eq:flow_reserver}
%&f^m_{i,j-1}+ f^m_{i,j+1}+ f^m_{i+1,j}+ f^m_{i-1,j} = r^m_{i,j}
%\end{align}
\begin{equation}
  \label{eq:diff_flow}
c_{i,j}-(f^m_{i,j-1}+ f^m_{i,j+1}+ f^m_{i+1,j}+ f^m_{i-1,j}) = l^m_{i,j}.
\end{equation}

Accordingly, the path generation problem (\ref{eq:ilp_1})--(\ref{eq:ilp_2}) can be 
relaxed as
\begin{align} 
  \text{minimize} &\quad \alpha \cdot \sum_{m=1}^{n_p}p_m + 
  \beta \cdot \sum_{m=1}^{n_p} \sum_{\mathtt{C}_{i,j} \in \mathbf{C}} l^m_{i,j} 
  \label{eq:ilp_1_ll}\\
\text{subject to} & \quad (\ref{eq:valve_cell})-(\ref{eq:flow_var}), 
(\ref{eq:valve_on_path}),  \and
(\ref{eq:disjoint_loop_violation})-(\ref{eq:diff_flow})
\label{eq:ilp_2_ll}
\end{align} 
%where $\alpha$ and $\beta$ are positive constants balancing the objectives.
where $\alpha$ is set to 10 times of $\beta$ to give priority to the 
lower number of test patterns.

Since the ILP model (\ref{eq:ilp_1_ll})--(\ref{eq:ilp_2_ll}) only minimizes
the occurrence of disjoint loops, such loops may still appear in the result returned
by the solver. An example of this case is shown in \figname~\ref{fig:minimum_loops}(a). 
To cover the cells on the disjoint loop, we split these cells into two parts.
The first part is merged with the original simple path to construct a path 
from the pressure source to the pressure sensor, as shown in  
\figname~\ref{fig:minimum_loops}(b), where two valves are left uncovered. These
valves are merged into a new path modified from the original simple path to
increase fault coverage, as shown in \figname~\ref{fig:minimum_loops}(c). In
implementing this post-processing step, we break all the disjointed loops resulted by
solving (\ref{eq:ilp_1_ll})--(\ref{eq:ilp_2_ll}) and merge them into existing
simple paths together to reduce the number of newly created test paths.

\subsection{Constructing cut test patterns}\label{sec:cut}

After generating the test paths, we also need to create cuts to test whether
all valves can be closed. In this test, if all valves in a cut are closed
correctly, there should be no test pressure going from the pressure source to
the sensor. Because a cut separates the pressure source and the pressure sensor, 
the two ends of a cut must touch the boundary of the chip, as
shown in \figname~\ref{fig:path_cutset}(b).
%Otherwise, there should be a valve that cannot be closed to allow the pressure
%to pass through. 
Observing this phenomenon, we search the valves along the
boundary of the chip in two directions starting from the pressure source 
until the pressure sensor is reached from both directions.
%, as shown in \figname~\ref{fig:path_cutset}(d). 
The resulting valves are added into two sets, and 
%Consequently, we can find two sets of valves and 
a cut must include a valve from each of these sets. 
Therefore, the problem to find cuts can be reformulated as follows:

\textit{Cut generation}:  Find a minimum number of simple paths without loops
to form cuts that cover each valve in the chip at least once.  Each cut must
have a valve from each of the two valve sets identified by searching along the
external boundary of the chip from the pressure source to the pressure sensor.
%as shown in \figname~\ref{fig:path_cutset}(d).

The problem above is a complementary problem of finding a set of test paths
covering all valves described in Section~\ref{sec:flow_paths}.  To solve this
problem, we can still use the formulation
(\ref{eq:ilp_1_ll})--(\ref{eq:ilp_2_ll})
together with the constraint that a cut must contain a valve in each of the
two sets found by searching along the chip boundary. 
In this complementary formulation, constraint variables are not assigned to cells 
but to the obstacle blocks between valves to form cuts.
%which block all paths from the pressure source to the pressure sensor. 
This variable assignment is illustrated in \figname~\ref{fig:cut_var}.

%\begin{figure}
%{\figurefontsize
%\centering
%\input{Fig/cutset_model.pdf_tex}
%\caption{Cut-set constraint variables and control leakage fault test. (a) Constraint
%variables for cut-set modeling. (b) Control leakage test scenarios.}
%\label{fig:cutset_model}
%}
%\end{figure}
  
As discussed in Section~\ref{sec:test_strategy}, we must prevent the
scenario shown in \figname~\ref{fig:path_cutset}(c) and (d) from happening.
 Otherwise, two faulty valves mask each other and thus 
cannot be detected.  This scenario can be described as that a new
cut can be formed by only one new valve with other valves from an existing 
cut.  Assume there is a valve $\mathtt{V}_{i,j}$ and
%at the location $(i,j)$ and 
the two obstacle blocks at the two ends of the valve are indexed by $(i_1,j_1)$ and
$(i_2,j_2)$.  To prevent fault masking as illustrated in \figname~\ref{fig:path_cutset}(c)
from being formed, we add an additional constraint to the optimization problem
as 
\begin{equation}
c^m_{i_1,j_1}+c^m_{i_2,j_2}-1\le v^m_{i,j}.
\end{equation}
%\text{where $c_i$ and $c_j$ are the two obstacle space at the two ends of the $k$th vavle.}
This constraint informs the solver that if the two ends of a valve are in 
a cut, this valve must be included in the cut. 
%For example, the valve at the location $(i-1, j)$ in
%\figname~\ref{fig:cut_var} must be included in the cut if the two
%ends of the valve are a part of the cut, indicated by 
%$c^m_{i,j}=1$ and $c^m_{i-2,j}=1$.
Consequently, it is not possible anymore that another single valve forms a new cut
with some valves in the current cut, so that the single-fault masking problem
can be avoided. 

%\hspace*{15pt}
\subsection{Testing control layer leakage}
\label{sec:control_layer_test}
\begin{figure}
{\figurefontsize
  \begin{minipage}[b]{0.20\textwidth}
    \centering
    \input{Fig/control_test.pdf_tex}
    \caption{Partial control leakage test with paths.}
    \label{fig:control_test}
  \end{minipage}
  \hspace{14pt}
  \begin{minipage}[b]{0.25\textwidth}
\centering
\input{Fig/long_channel.pdf_tex}
\caption{Test path through a long channel twice. The missing valve causes
bypassing of valves on the path. }
\label{fig:long_channel}
  \end{minipage}
}
\end{figure}

If there is a leakage from a control channel to another control channel,
an air pressure in this control channel 
to close a valve also closes the other valve simultaneously. 
Such faults usually happen at adjacent valves during manufacturing
\cite{HuYHC14}.
%In the proposed method, we test potential leakage faults between the valves
%surrounding a cell. This formulation can be extended easily to test leakage
%problems in the control layer within a large area, though inevitably the
%computational complexity will also increase.
%To test whether two neighboring valves are always closed at the same time, 
%one valve can be closed by an air pressure in its control channel 
%and the other can be opened by releasing the air pressure in its control channel. 
%
%In addition, a cut covering these two valves
%are created to separate the source and the sink on the chip.
%In the test, if the test pressure can still be detected at the sink,  
%a control layer leakage problem is found.
%

In the test paths described in Section~\ref{sec:flow_paths}, 
control layer leakage has actually been covered in part. For example, 
on the test path in \figname~\ref{fig:control_test}, the valves that
form the boundary of the path are closed while the valves along the path are
opened. If the control pressure to a closed valve at the boundary of the path 
leaks into the control channel of an opened valve on the test path, 
the test path is then blocked. Consequently, no test pressure 
can be detected by the sensor, indicating a fault in the chip. 

Besides the valve pairs above, there are further valve pairs whose control channel
leakage should be tested. For example, the valves 
along the path in \figname~\ref{fig:control_test} need to be checked,
because these valves are opened together during the test, so that the case a
valve is closed and another is opened is not covered. To deal with these
untested valve pairs, we generate further paths going through only one of the valves
in an untested valve pair along a path, while the other valve is kept closed. 
In other words, only one of the valves in such a pair appears on a new test path. 
%while the other is on the boundary of the path.   
%Similar to constructing test paths described in
%Section~\ref{sec:flow_paths}, the number of these new test paths should be
%minimized to reduce test cost.

To generate new test paths to test control channel leakage, we first scan
existing flow paths generated as described in Section~\ref{sec:flow_paths} to
identify untested valve pairs.  
%For each of such valve pairs, we keep one valve
%open and the other closed. 
%Since the
%state of the opened valves are tested by flow paths, 
%we revise the flow path problem in Section~\ref{sec:flow_paths} with an
%additional constraint describing the states of a valve pair during the test.
Assume that the valves at $(i_1,j_1)$ and $(i_2,j_2)$ 
%are identified 
need to be
tested for control channel leakage. We force one of them to stay open and the other to be
closed when new test paths are generated. As defined in 
Section~\ref{sec:flow_paths}, the 0-1 variable $v^m_{i,j}$ represents whether a
valve is on the $m$th path. Therefore, the additional constraint 
can be written as
\begin{equation}
v^m_{i_1,j_1}+v^m_{i_2,j_2}=1
\end{equation}
which can be included in the optimization problem
(\ref{eq:ilp_1_ll})--(\ref{eq:ilp_2_ll}) to cover control leakage.

%The control layer leakage problem is solved in each subblock in the
%hierarchy. Thereafter, there are still a few valve pairs at the boundary
%between two subblocks that cannot be included in the control leakage test
%because the flow path problem is formed for each subblock individually. These
%remaining valve pairs are connected using a variant of breadth-first search to form the
%test paths.

\begin{figure}
{\figurefontsize
\centering
\input{Fig/supper_cell.pdf_tex}
\caption{Generating test paths on an FPVA with a long channel. (a) The original
  FPVA. (b) Converting the FPVA into a connection graph. (c) Cells
  in a long channel are collapsed into a super cell. Flow paths are constructed on
  the new connection graph. (d) A corresponding flow path on the original
FPVA.}
\label{fig:super_cell}
}
\end{figure}

\subsection{Dealing with long channels and obstacles}\label{sec:walls_holes}

An FPVA may contain long channels and obstacles to enable certain
functionalities such as allowing special devices to be built into the chip. 
%as illustrated in \figname~\ref{fig:dynamic_devices}(d). 
In long channels, some valves are missing. 
%In the locations occupied by obstacles, no valves are
%built, equivalent to the case that valves can never be opened.
%has shown such long channels and obstacles
%made up by missing valves or the valves that always closed. 
When constructing a test path, all the valves on the path are opened
and the other valves are closed to form the boundary of the path. However,
missing valves and obstacles may cause special problems in the test.
%cannot open valves occupied by obstacles when generating cuts.
For example, the flow path in \figname~\ref{fig:long_channel}  runs through
a long channel twice.  Since the missing valve in the long channel cannot be
closed, the short path through the missing valve leaks the test pressure.
Accordingly, the sensor can still detect a test pressure, even if some
valves 
%on the segments of the test paths 
on the left side of the long channel
cannot be opened due to manufacturing defects.
Consequently, the test path become invalid, although if meets the constraints
described previously.

%\begin{figure}
%{\figurefontsize
%\centering
%\input{Fig/long_channel.pdf_tex}
%\caption{A flow path runs through a long channel twice. The missing valve causes a leakage. }
%\label{fig:long_channel}
%}
%\end{figure}

To address the issues above, we introduce the concept \textit{super cell}.
\figname~\ref{fig:super_cell}(a) is the FPVA with a long channel. In
\figname~\ref{fig:super_cell}(b), the FPVA is converted to a connection graph
to show the relationship between cells and valves, where nodes represent the
cells and edges represent the valves. Because the cells in
the long channel are always connected together, we can collapse them into one
super cell. All the valves surrounding the long channel then surround this new
super cell. The new connection graph of is shown in
\figname~\ref{fig:super_cell}(c). We can then apply the ILP model
(\ref{eq:ilp_1_ll})--(\ref{eq:ilp_2_ll}) 
%(\ref{eq:ilp_1})--(\ref{eq:ilp_2}) 
on the new connection graph to construct test paths.  Because all the cells
inside a long channel are treated as one cell, the corresponding paths do not run
through the long channel more than once. \figname~\ref{fig:super_cell}(d)
shows one of the actual test paths on the original FPVA. 
%This path can
%test whether all valves on it can be opened reliably. 
%Similarly, obstacles can also
%be collapsed into super nodes 
%%and considered as valves that can never be opened,
%%so that the construction of cuts can be conducted directly on the modified
%%connection graph.
%to construct cuts efficiently.

The concept of collapsing nodes in the connection graph
is also applied to generate cuts while considering obstacles 
efficiently.  Instead of representing cells, 
the nodes in the connection graph then represent the small obstacle blocks between valves
as shown in \figname~\ref{fig:cut_var} to cut off all paths from the pressure
source to the sensors. 


