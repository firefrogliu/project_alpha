%\clearpage
\section{Requested modules/funds}
\subsection{Basic Module}
\subsubsection{Funding for Staff}
\vskip 8pt
\hskip 10pt $\bullet$\hskip 5pt  One full-time research assistant (PhD
candidate or postdoctoral researcher), TV-L E13, for \\
\phantom{a} \hskip 12pt three years
\vskip 2pt

Since this is an interdisciplinary research topic, an extensive understanding
of design automation for integrated circuits and 
%the field of 
biochemical
systems is required. In addition, the project 
provides a good chance for the EDA research community
to reach out to other research fields.
Accordingly, it is applied that this position 
%is applicable to 
can be filled by a postdoctoral researcher.

\vskip 8pt
\hskip 10pt $\bullet$\hskip 5pt  
Two work student positions (each 40 hours/month) for three years 
%are needed:
%\newline Wiss. Hilfskraft: 19.50h: (\EUR{984.00} Gehalt + \EUR{212.92} AG-Anteil )
%$\times$ 33 months ~
\vskip 2pt

This interdisciplinary research project plans to develop a design
automation framework for microfluidic biochips. It requires that 
many algorithms from other research fields, e.g., design automation for 
integrated circuits, are implemented and tested, while
considering the unique characteristics of biochips. 
%Accordingly, this project requires
%%work involves much
%a lot of programming and test of existing algorithms, so that 
Therefore, the support from work
students is highly expected. Reciprocally, this project provides students 
a chance to expose themselves to various problems ranging from
the basic mechanism of biochips to high-level programming techniques.
%\textit{Therefore, two work student position (20 hours/week) for work package WP1--WP6
%for a total of 36 months are needed:
%\newline Wiss. Hilfskraft: 19.50h: (\EUR{984.00} Gehalt + \EUR{212.92} AG-Anteil )
%$\times$ 33 months ~= \EUR{39,498.00}
%}


\vskip 12pt
\subsubsection{Direct Project Costs}

\paragraph{Equipment up to \EUR{10,000}, Software and Consumables}
None
\vskip 8pt

\paragraph{Travel Expenses}
Trips for presenting research results at national and international
conferences are planned.

Three trips to overseas conferences, e.g., ACM/IEEE Design Automation
Conference, USA, IEEE/ACM Conference on Computer-Aided Design, USA:
\EUR{3,000.00}$\times$3

Three trips to European conferences and workshops:
\EUR{1,500.00}$\times$3

Total sum of travel expenses: \EUR{13,500.00}


\vskip 12pt
\paragraph{Visiting Researchers}
None
\vskip 8pt
\paragraph{Expenses for Laboratory Animals}
None
\vskip 8pt
\paragraph{Other Costs}
None
\vskip 8pt
\paragraph{Project-related publication expenses}
Journal paper publication with IEEE open access option: 
\EUR{1,900.00}

\vskip 12pt
\subsubsection{Instrumentation}
\paragraph{Equipment exceeding Euro 10,000}
None
\vskip 8pt
\paragraph{Major Instrumentation exceeding Euro 50,000}
None
\vskip 8pt
\subsection{Module Temporary Position for Funding}
None
\vskip 8pt
\subsection{Module Replacement Funding}
None
\vskip 8pt
\subsection{Module Temporary Clinician Substitute}
None
\vskip 8pt
\subsection{Module Mercator Fellows}
None
\vskip 8pt
\subsection{Module Workshop Funding}
None
\vskip 8pt
\subsection{Module Public Relations Funding}
None
\vskip 8pt

%\vskip 15pt
%\clearpage
\section{Project requirements}
\subsection{Employment status information}
%Dr.-Ing. Bing Li, 
PD Dr.-Ing. habil. Bing Li,
Akademischer Rat auf Zeit bis 31.07.2021,
%Lehrstuhl f\"ur Entwurfsautomatisierung,
%Fakult\"at f\"ur Elektrotechnik und Informationstechnik, \\
Institute for Electronic Design Automation,
\tum

%Wissenschaftlicher Mitarbeiter, 
%current contract to 30.09.2018 (extension
%expected), supported by BMBF


\vskip 12pt
\subsection{First-time proposal data}

This is the first-time proposal from PD Dr.-Ing. habil. Bing Li.
%Dr.-Ing. Bing Li.
\vskip 12pt

\subsection{Composition of the project group}

Besides the research assistant supported by this project, other members in
the research group on emerging systems in the Institute for Electronic Design
Automation, TUM, will work on related topics ranging from test of microfluidic
biochips to optical interconnect systems. These members include Prof. Ulf
Schlichtmann and
%Mr. Chunfeng Liu (PhD candidate, supported by TUM IAS, 75\% TV-L E13), 
Ms. Ying
Zhu (PhD candidate, supported by TUM IGSSE scholarship). These members will provide necessary
technical support to this project, 
specially the knowledge of existing design methodologies
and algorithms.  

%\begin{tabular}{llll}
%Name & academic title & employment status & type of funding\\
%Bing Li          &  Dr.-Ing.  &   TV-L E13 & Landesmittel  \\
%%Tsun-Ming Tseng  &  MSc.      &?    &   ?\\
%Chunfeng Liu     &  MSc.      & 75\% TV-L E13   &   TUM IAS\\
%Ying Zhu         & 
%\end{tabular}

\subsection{Cooperation with other researchers}
\subsubsection{Researchers with whom you have agreed to cooperate on this
project}

%In this project, the Institute for Electronic Design Automation (EDA) at TUM
%will collaborate with the research group of Prof. Krishnendu Chakrabarty from
%Duke University, USA, and the research group of Prof. Tsung-Yi Ho from
%National Tsing Hua University, Taiwan. 
%Prof.  Chakrabarty is  a leader in the field of microfluidic biochips and 
%microbiology applications. His research ranges from application 
%mapping to manufacturing testing, which has resulted in more 
%than 550 papers published at top conferences and journals,
%with many highly recognized best paper awards such as 
%the IEEE Transactions on
%Computer-Aided Design of Integrated Circuits and Systems
%Donald O. Pederson Best Paper Award in 2015.
%%He has served as the Editor-in-Chief of 
%%ACM Journal on Emerging Technologies in Computing Systems,
%%2010--2012 and IEEE Design \& Test of Computers, 2010-2012, and is serving as
%%the Editor-in-Chief of the IEEE Transactions on Very Large 
%%Scale Integration Systems.
%His expertise on design, test and microfabrication will provide us an
%extensive support during this project.
%Prof. Tsung-Yi Ho is a leading scientist in microfluidic integration.
%%He has been the recipient of the Humboldt Research
%%Fellowship (2012--2013), which he spent in TUM. He has
%%also received the Hans Fischer Fellowship from TUM-IAS, 
%His research focuses on cyber-physical integration of biochips and he
%received the Best Paper Award at the VLSI Test Symposium (VTS)
%in 2013. 
%%Currently he is serving as an ACM Distinguished Speaker, 
%%Associate Editor of the ACM Journal on Emerging Technologies in Computing
%%Systems and the IEEE Transactions on Computer-Aided Design of Integrated
%%Circuits and Systems.
%His experience in system integration of microfluidic biochips will be a very
%important input to this project.
\vskip 2pt
Prof. Krishnendu Chakrabarty, Duke University, US\\
Prof. Tsung-Yi Ho, National Tsing Hua University, Taiwan
\vskip 8pt


%%The EDA institute is also collaborating with Fraunhofer EMFT in Munich, where
%%industrial projects on microfluidic biochips are a key area of research and 
%%development. 
%%During the project period and
%%further, we will hold regular meetings, discussions, and potentially joint
%%projects to evaluate our
%%algorithms and design ow on industrial designs.
%The EDA institute is also cooperating actively with 
%%In addition, we also work actively with  
%several other researchers on
%modeling and analysis of reliability effects within TUM 
%(e.g., with Prof. Andreas Herkersdorf),
%within Germany (e.g., with FZI and FAU) 
%and internationally (e.g., with CMU, USA; NTU, Singapore). These
%interactions and collaborations will positively 
%influence the work proposed in this project.


%\vskip 12pt
\subsubsection{Researchers with whom you have collaborated scientifically
within the past three years}
Within several projects in the past,
%the institute for Electronic Design Automation
we have collaborated with
%with national and international 
other researchers extensively.
%from universities, research institutes and industry.
To be named are the strong collaborations with 
%Prof. Andreas Herkersdorf (TUM), 
%Prof. Samarjit Chakraborty (TUM), 
%Prof. Norbert Wehn (TU Kaiserslautern), 
Prof. Krishnendu Chakrabarty (Duke University, US),
Prof. David Z. Pan (UT Austin, US), 
Prof. Jiang Hu (Texas A\&M University, US),
Prof. Yiyu Shi (University of Notre Dame, US),
Prof. Paul Pop (Technical University of Denmark),
Prof. Bhargab B. Bhattacharya (Indian Statistical Institute, India),
Prof. Masanori Hashimoto (Osaka University, Japan),
Prof. Tsung-Yi Ho (National Tsing Hua University, Taiwan), 
Prof. Hailong Yao (Tsinghua University, China).

%Prof. G\"unhan D\"undar (Turkey), 
%Prof. Xing Zhou (NTU, Singapore), Prof. Davide Bertozzi (U. Ferrara, Italy), 
%Prof. Wolfgang Ecker (Infineon), 
%Dipl.-Ing. Georg Georgakos (Infineon), 
%Dr. Helmut Reinig (Intel Mobile Communications) 
%Dr. Veit Kleeberger (Infineon),
%and Dr. Sani Nassif (Radyalis, US).


\subsection{Scientific equipment}
The Institute for Electronic Design Automation is equipped with a central file
server and a cluster of computing servers containing 
Dual-/Quad-Core CPUs running Linux. Therefore,
all necessary equipment for running the project is already available.

\subsection{Project-relevant cooperation with commercial enterprises}

None

\subsection{Project-relevant participation in commercial enterprises}

None

\vskip 2pt
\section{Additional information}
None



