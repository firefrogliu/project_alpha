%\clearpage
\section{Objectives and work programme}
\subsection{Anticipated total duration of the project}
%36 months
Three years

\vskip 10pt
\subsection{Objectives}

\revadd{The major objective of this project is to 
develop computer algorithms for design automation of microfluidic biochips. Based
on fundamental components such as valves and devices and design constraints, 
efficient microfluidic architectures 
will be synthesized automatically. The resulting design framework will be capable of offloading
design tasks from chip designers by computer algorithms
and assisting them to explore new chip designs efficiently. }


From the perspective of computer-aided design,
synthesis of microfluidic biochips can be partitioned 
into three levels. At
the lowest level, valve activities and flow transportation are investigated. 
%to generate an efficient chip architecture. 
The task of this step % micro-synthesis aims
is to generate an efficient chip architecture for a given bioassay or
multiple bioassays automatically. 
The second level of synthesis deals with special requirements of washing and
flow-/control-layer codesign. The result of this synthesis is overlaid upon
the biochip architecture generated in the previous step to form a complete
architecture-schedule solution. Since the first two steps focus on concrete
behavior of basic components such as valves and channels, they are called
\textit{microsynthesis} henceforth.
Biochips considered in microsynthesis are 
\begin{wrapfigure}[22]{r}{0.53\textwidth}
{
\vskip 2pt
\figurefontsize
\centering
\input{Fig/task_relation.pdf_tex}
\caption{Synthesis levels for biochips.} 
\label{fig:task_relation}
}
\end{wrapfigure}
relatively small-scale due to
the scalability challenges in optimizing exact switching activities of all
valves and flow transportation tasks. 
These biochips %of this scale 
are very useful in small biochemical labs or in the scenarios such
as point-of-care test. 

With the advance of manufacturing technology, 
the integration density of biochips has increased tremendously.
%For example, recent manufacturing technologies have enabled a
%valve density reaching 1 million per cm$^2$ %\si{\square\centi\metre}
%\cite{C2LC40258K}. 
Correspondingly, a bioassay executed by a biochip of this scale also
contains a very large number of operations for resource-intensive
applications, e.g., exhaustive diagnoses for illness detection.
Therefore, synthesis of biochips at the highest level, or macrolevel, 
thus called \textit{macrosynthesis} henceforth,
deals with the challenges considering a large-scale system integration.
%are investigated. 

The synthesis techniques of the three levels form a comprehensive tool chain
for flow-based biochips. %At the last stage of synthesis, 
%Afterwards, the
%A generated biochip and its execution schedule are integrated into
%a cyber-physical system. Sensors and/or cameras in such a system
%are used to assist the execution of bioassays, specially those 
%with non-deterministic execution duration, and
%In addition, this
%cyber-physical integration also 
%to enable innovative test methods to detect 
%manufacturing 
%faults.  
The relation of these tasks from architectural synthesis to cyber-physical
integration is illustrated in
\figname~\ref{fig:task_relation}.
\revadd{Since it is challenging for designers to manually deal with these tasks for
complex microfluidic systems, the goal of this project is thus to develop design 
automation algorithms to assist designers to deal with them. 
This scenario is similar to the case that EDA research contributes in 
the general scope of the semiconductor industry, where
%. Although the 
%EDA companies do not build ICs directly, 
it assists designers with software tools to manage 
the exploding design details of complex
ICs.}

In this project, we will focus on micro- and macrosynthesis of biochips to
generate efficient biochip 
%architectures 
designs
automatically. 
The unique characteristics of biochips discussed in the previous section 
will be incorporated into this synthesis process. Furthermore, the interaction
of the automatically generated biochip 
%architectures 
designs
with sensors and cameras in
cyber-physical systems will be investigated.
%together with our
%collaboration institutes.
This %interdisciplinary 
project will demonstrate the effectiveness of 
design automation algorithms in improving the performance and reliability of
biochips. More importantly, it will bridge microfluidic technologies 
with real-world applications using the developed tool chain to realize the great potential of
microfluidic biochips. 
%will be unleashed to benefit the everyday life of human beings.
 
\subsubsection{Microsynthesis of microfluidic biochips}



%Synthesizing a biochip aims to generate a chip architecture that
%can execute a bioassay efficiently. 
%The major steps of this synthesis process are
%shown in \figname~\ref{fig:biochip_synthesis}. 
Since microsynthesis deals with compact 
biochips that only contain a small number of devices,
%During micro-synthesis, 
%the number of devices in a biochip
%executing the operations in an assay 
%is usually not large due to cost,
%much smaller than the number of operations, 
operations need to be arranged along the timeline to reuse devices.
Consequently, operations are assigned in queues at devices, such as the schedule in
\figname~\ref{fig:biochip_synthesis}(a). 
%To determine device assignment and
%time slots for operations, or operation scheduling, 
%%This operation scheduling 
%is very similar to task scheduling in high-level synthesis 
%for integrated circuits \cite{Micheli94}. 
%From the scheduled tasks known, 
With this schedule,
a biochip architecture is generated automatically that
carries out the operations and the transportation tasks 
between devices.
%efficiently. 
To reduce the execution time of the bioassay, 
operations should
not wait 
\begin{wrapfigure}[33]{r}{0.4\textwidth}
{
%\vskip -2pt
\figurefontsize
\centering
\input{Fig/schedule_storage_p.pdf_tex}
\caption{Storage reduction. (a) Sequencing graph. 
(b) Schedule with two storage requirements. (c) Schedule
with one storage requirement. The execution 
times of the bioassay with these two schedules are equal.}
\label{fig:schedule_storage_p}
}
\end{wrapfigure}
at devices unnecessarily longer than required. In addition,
transportation tasks should affect the execution of operations as little as
possible by moving fluid samples to target devices on time. 
%that can execute the
%schedule and fulfill the transporation tasks between devices is
%derived. 
%For the architecture, 
%%In the last step, 
%a physical design containing the locations of
%devices and the connection of transportation channels 
%can be produced as the input data for manufacturing.


%\clearpage
\subsectitle{Storage minimization during scheduling}

%Synthesizing biochips, however, requires a storage model different from the one in
%high-level synthesis for integrated circutis. 
When devices in a biochip execute operations according to a schedule, the resulting
fluid samples of some operations may not be used immediately. 
Consequently, these fluid samples need to be stored in the chip temporarily. 
%Since fulid samples have a physical volume, 
%intemediate results of operations should moved to 
%be allocated some space
%on the chips 
%should be stored temporarily on the chip before they are
%used as inputs of further operations. 
The durations of these storage requirements affect %the size of 
the overall execution time of the bioassay, because improper storage may block the
transportation of other fluid samples. In addition, a large number of
simultaneously stored fluid samples consume many resources and 
require a large area
due to their physical volume.

Assume that the pure transportation time from a device to another device 
%or the storage channel 
in a biochip is a constant $u_c$. 
If an intermediate fluid sample needs to wait for a time larger than $u_c$ before
it is used, it must 
%schedule 
%produces a transportation time larger
%than $u_c$, the corresponding fluid sample 
be stored somewhere in the chip temporarily. %before it is  used. 
Figure~\ref{fig:schedule_storage_p}(b)--(c) illustrates
two schedules of executing the bioassay in \figname~\ref{fig:schedule_storage_p}(a)
%with five operations 
using two mixers.
In \figname~\ref{fig:schedule_storage_p}(b),
operation $O_2$ is scheduled before $O_3$. 
Since the result of $O_2$ must be stored until it is used by $O_4$ and $O_5$, 
half of the mixing result of $O_2$ 
needs to be transported to a location for storage $O_2\to O_4$, which starts $u_c$
time after $O_2$ finishes.
Thereafter, the other half of the mixing result is %similarly  
moved out 
%of Mixer2 
to initiate storage $O_2\to O_5$.
Consequently, two storage requirements are generated from this schedule. %lasting six time units in total. 
Since the lifespans of 
these storage requirements 
overlap, the storage capacity
of the biochip should be at least two units.
To reduce the storage requirement, $O_3$ can actually be scheduled before $O_2$ as
shown in \figname~\ref{fig:schedule_storage_p}(c). In this schedule, half of 
the result of $O_2$ is 
transported to Mixer1 and $O_5$ starts after the storage $O_3\to O_5$ is
fetched into Mixer2. This improved schedule generates
only a single storage requirement, %lasting a shorter time.  
while the overall execution time of the bioassay is not changed. 
%four time units. 
%Although these two schedules lead to the same execution time for the assay, 
%the additional storage requirements in \figname~\ref{fig:schedule_storage}(b) 

In this example, we can observe that
%should be met by more resource built in the chip. 
the lifespan of a stored fluid sample is determined by the difference between
the ending time of the parent operation $O_i$ and the starting time of the child
operation $O_j$, denoted by $u_{i,j}$,  e.g., $u_{2,4}$ in \figname~\ref{fig:schedule_storage_p}(b).
%
To reduce storage requirements, we introduce a \textit{storage
minimization objective} and minimize the intermediate storage requirements
when scheduling operations to devices by solving
%
%the total time of all storage requirement together with the minimum execution
%time of the assay by solving the ILP problem below.
\begin{align} 
\text{\textit{Minimize}} & \quad \alpha t^E+\beta\sum_{e_{ij}\in
\mathcal{E}\bigwedge d_i\ne
d_j }u_{i,j} 
 \label{eq:min_storage_obj}
%\text{subject to} & \quad
%\text{(\ref{eq:uniqueness})--(\ref{eq:latest_finishing}})\label{eq:op_cond}
\end{align}
%where $\sum_{\substack{(o_i,o_j)\in O\\d_i\ne d_j  }}(u_{i,j}-u_c)$ is the
%total storage
%time in the schedule. 
where $t^E$ is the total execution time of the bioassay similar to the
definition in
\cite{Micheli94}, %execution time and storage requirement minimization,
$\mathcal{E}$ is the set of the edges in the sequencing graph, and 
$e_{ij}\in \mathcal{E}$ limits storage minimization to operation pairs with a direct
edge in the sequencing graph.
$d_i$ and $d_j$ represent to which devices operations $O_i$ and $O_j$ are
assigned, respectively. Therefore, $d_i \ne d_j $ excludes the operation pairs assigned to the same device
where no transportation is needed.
The two constants $\alpha$ and $\beta$ 
control the priority of the optimization objectives. The other
constraints of this optimization problem are the same as those defined 
for high-level synthesis of digital circuits \cite{Micheli94}. 
After solving (\ref{eq:min_storage_obj}), 
a schedule
like \figname~\ref{fig:schedule_storage_p}(c) instead of 
\ref{fig:schedule_storage_p}(b) is generated as the input to architectural
synthesis in the next step.
%The challenge of this task is that solving (\ref{eq:min_storage_obj})
%directly might be very time-consuming and efficient heuristic techniques
%need to be developed accordingly.

\revadd{During scheduling, if no device reuse is allowed in the biochip, 
a device can only execute one
operation. The technique above, however, can still be applied 
to determine the relative execution relation of operations more efficiently
than manual design.}

\subsectitle{Architectural synthesis with distributed channel storage}
\label{sec:arch_syn}

The schedule of a biochip specifies when and by which devices operations are executed. 
In addition, it states
when and where transportation requirements between devices appear.
%after operations are executed. 
To carry out operations and fulfill transportation tasks efficiently, 
the relative locations of devices as well as the
channels connecting them should be determined. 
%to conduct these transportation tasks.
\revadd{The types of the devices, e.g., mixers, heaters and filters, correspond to the
types of the operations to be executed. These types are specified by the
designer of the bioassay but
%have no influence in 
%determining the placement of the devices, 
only the dimensions of the devices influence their placement and the following
routing of transportation channels.
%because the relation of flow transportation 
%affects the relative locations of the devices.
}

When fluid samples are moved along channels, they can stop the movement 
and stay in a channel segment temporarily to avoid a transportation conflict,
as cars stop before a red traffic light to let cars of the other direction 
pass. Namely, fluid samples can be cached in any channel in the biochip, 
provided that the channel is not used for other transportation tasks. 
After the conflict is resolved, fluid samples can resume the movement  
toward target devices. This \textit{distributed channel storage} mechanism,
or \textit{caching}, is a unique characteristic of biochips
%significant difference 
compared with %the storage units RAM in 
electronic systems, where data must be stored in memory units such as RAM instead
of on interconnect wires.  

This distributed caching mechanism allows a high efficiency of storing and fetching of
fluid samples by avoiding access congestion at the ports of a dedicated storage
unit. In addition, accessing on-the-spot distributed channel storage 
is more convenient because fluid samples can be cached close to the devices. 
Furthermore, this feature enables a chip architecture with
a unified representation of transportation and storage channels to
%with which channel efficiency can also be improved significantly since 
improve channel efficiency. 
%we can move fluid samples along transportation channels and stop them when there is a
%transportation conflict, just as in a road transportation system.


%When a large assay is executed by several devices, 
%transportation channels need to be built between nearly
%any pair of devices to move fluid samples between devices efficiently.
Since the flow layer in a biochip is only two-dimensional, 
intersections between flow channels in a complex system
usually cannot be avoided. At each intersection, a switch needs to be built 
to direct the transportation flow to the target device. 
A switch is constructed by four valves at an intersection, 
as shown in \figname~\ref{fig:switch_grid}(a). At any
given moment, two out of the four valves in a switch are opened to 
connect two channel segments.  Consequently, 
a fluid sample needs to travel through several channel segments connected by
switches to reach the target device. 
Correspondingly, these channel segments form a  
%a transportation channel between two devices becomes a path,
%formed by several channel segments connected by switches. 
%Such a path is 
\textit{transportation path} between two devices.

%Besides channels, the locations of devices should also be determined.
%These locations should be assigned together with the construction of 
%transportation channels because the distance and relative locations of devices
%affect how channels are constructed and how they intersect with each other.

Considering devices and channels together, the architecture of a biochip can
be described as devices connected by a network of channel segments. %in the form of a grid. 
%This is different from normal IC design because crossing is allowed in this
%architeture, but the channels crossing the same node should not be used at
%the
%same time to avoid contamination. 
For example, the architecture of a biochip 
with five devices is 
shown in \ref{fig:switch_grid}(b), where solid nodes
represent switches and circles represent devices. 
Transportation paths between devices are
formed from channel segments and connected by switches. %such as path 1 and path 2.
%in \figname~\ref{fig:switch_grid}(b).
Since transportation paths are used only when there is a fluid sample
traveling through them, channel segments can be reused by
different paths to increase efficiency. %of channel segments can be increased.
For example,  path 
1 and path 2 in \figname~\ref{fig:switch_grid}(b) 
share a common channel segment which they use at different times.



%With transportation paths formed from channel segments, 
On the network of channel segments,
distributed channel storage can be formed to replace
%In addition, the fluid storage need also be implemented in the
%chip. 
the dedicated storage unit in \figname~\ref{fig:biochip_arch}, which
suffers access congestion at its ports. 
%
%Observing that the storage cells 
%inside a dedicated storage unit
%%that store fluid samples 
%are in fact channel segments, we distribute the fluid storage
%directly in channel segments. 
For example, in
\figname~\ref{fig:switch_grid}(c), along path 3 a fluid sample is moved to the
channel segment between A and B. However, 
the next operation using this 
fluid sample is scheduled later, so that the fluid sample must stay in the
channel segment. During the lifespan of this storage, the channel segment
between A and B cannot be used by other paths 
and the valves at the two ends of this channel segment must be closed.
Consequently, other transportation tasks between devices 
need to be fulfilled by paths that 
do not include this channel segment, 
such as path 4 and 5 in \figname~\ref{fig:switch_grid}(d).
When the cached fluid 
%
\begin{wrapfigure}[33]{r}{0.40\textwidth}
{
%\vskip 8pt
\figurefontsize
\centering
\input{Fig/switch_grid.pdf_tex}
\caption{Switch and channel storage. Circles represent devices and solid
nodes represent switches.  (a) Switch structure. (b) Two transportation paths
sharing one channel segment with time multiplexing. (c) Fluid sample to
channel storage. (d) Storage in channel segment. (e) Fluid sample from channel
storage to device. (f) Connection grid for architectural synthesis.}
\label{fig:switch_grid}
}
\end{wrapfigure}
%
sample is finally needed, it is moved to the target
device again by a newly 
constructed transportation
path, which is shown as path 6 
in \figname~\ref{fig:switch_grid}(e).

%Unlike the dedicated storage unit shown in
%\figname~\ref{fig:valve_mixer_storage}(c), the distributed storage in a
%channel segment has a higher access efficiency. 
%When a fluid sample stays in a channel segment,
%that segment is turned into a \textit{storage segment}. When the
%fluid sample moves again, the segment becomes a part of 
%the transportation path. 
The concept of 
%channel role switching %between transportation and storage
distributed channel storage
unifies transportation and storage.
%so that a dedicated storage unit is not required anymore. Consequently, the
%efficiency
%increases channel usage significantly. 
%because 
%the low-efficiency channels forming storage cells 
%are therefore excluded completely from the
%biochip. 
%to exclude the channel segments and valves 
%in the dedicated storage unit as shown in \figname~\ref{fig:biochip_arch}.
%In addition, this on-the-spot caching is closer to the target
%device than a dedicated storage unit, so that the execution efficiency of
%the bioassay can also be improved.
Therefore, we can
synthesize the architecture of a biochip from its schedule using
% requires to
%determine the relative locations of devices 
%as well as channel segments and the switches connecting them as the examples
%in  \figname~\ref{fig:switch}(b)--(e).  
%The exact locations of devices and switches are, however, not important 
%at this stage. 
%The devices, switches and their connections together are called 
%\textit{connection graph}. %nodes edges} 
%A valid connection graph should be capable of
%constructing all transportation paths specified in the schedule and caching
%intermediate fluid samples in channel segments. 
%To reduce resource usage, the synthesized connection graph 
%should contain as few edges as possible. 
%The connection graph is generated using 
a virtual \textit{connection grid} as shown in \figname~\ref{fig:switch_grid}(f). 
At each node on the grid, 
either a device or a switch can be assigned. An edge on the grid represents
a channel segment capable of caching a fluid sample.
%The transportation channels are contructed by the
%channel segments. 
%
%At each channel segment, we can cache one fluid sample. To
%determine on which nodes the devices are assigned and to construct the
%transportation channels, we establish a ILP formulation. In solving the
%optimization problem, we minimize the number of edges in the grid used by
%transportation channels, so that only the minimum connections are kept in the
%final result of architectural systhesis.
%
%In the connection grid, a node is denoted by $n_i$ and an edge is denoted 
%by $e_j$, where $i$ and $j$ are the node and edge indexes, respectively.
The task of architectural synthesis thus becomes to determine locations of
devices and build time-multiplexing transportation channels on the grid.
After architectural synthesis, all unused channel segments and switches in the virtual
connection grid will be removed to reduce chip area.

In architectural synthesis, we need to reduce resource usage 
while guaranteeing that the schedule can be executed correctly. Therefore, we 
describe the architectural synthesis task as an optimization problem, as
\begin{align} \label{eq:min_channel_obj}
&\text{\textit{Minimize}}  \quad \text{Number\_of\_channel\_segments}\\ 
&\text{s.t.}  \quad \text{Each device is assigned once on the grid;} \\
& \qquad\;\,  \text{Simultaneous transportation paths should } \nonumber\\
&\qquad\;\, \text{not intersect;} \label{non_intersetion_cons}\\
&\qquad\;\,\text{Channel storage should be isolated from}\nonumber\\
&\qquad \;\,\text{other paths;} \\
&\qquad\;\,\text{Scheduled operations are executed on time.}
\end{align}
%where $\sum_{\substack{(o_i,o_j)\in O\\d_i\ne d_j  }}(u_{i,j}-u_c)$ is the
%total storage

The non-intersecting constraint (\ref{non_intersetion_cons}) for a simple path, e.g., $p_r$ in
\figname~\ref{fig:switch_grid}, can be expressed as that the number of edges incident  
to a node on the path should not exceed two.
%specified by the number of edge incident to the nodes on a path.   
However, it becomes more complex when a storage is involved.
For example, the sub-paths $p_{r,1}$, $p_{r,2}$ and $p_{r,3}$
in \figname~\ref{fig:switch_grid}(f) represent the three phases of a channel
storage. While $p_{r,1}$ and $p_{r,3}$ can be considered as normal
transportation paths, $p_{r,2}$ should be processed specially because the two
ending nodes of the channel segment can be used by paths conducting other
transportation tasks, such as $p_r'$ in \figname~\ref{fig:switch_grid}(f). 

\revadd{In solving the optimization problem, the minimization of the number 
of channel segments built in the biochip also generates a compact architecture, because devices
and edges used to construct transportation paths tend to be close to each
other due to the optimization objective, no matter whether devices can be
reused in executing operations or not. Afterwards,
the locations of devices
are determined as indicated by the variables associated to the nodes. }

Since determining an optimal chip architecture on a connection grid is an
NP-hard problem due to the complex combinations of device locations and their
connections, heuristic strategies need to be deployed. For example, the
relative locations between devices can be preconfigured according to the
transportation requirements between samples produced by them. In addition,
nonregular connection grid structures may be explored to reduce the
computational effort. 
 
\subsectitle{Volume management and washing}
\label{sec:volume_washing}

\begin{wrapfigure}[15]{R}{0.32\textwidth}
{
\vskip -13pt
\figurefontsize
\centering
\input{Fig/volume_washing.pdf_tex}
\caption{Volume management and washing.}
\label{fig:volume_washing}
}
\end{wrapfigure}

A special characteristic of biochips is that volumes of fluid samples
should be managed during execution of a bioassay. For example, in
\figname~\ref{fig:sequencing_graph}, the mixer executing $O_{10}$ can
accept only half of the volumes from $O_7$ and $O_9$, and the other half needs to
be disposed of to the waste.
%Specially, 
This volume management should be dealt with very often in
diagnosis-purposed bioassays, where usually only the result of the
final detection operation is important. 
In practice, 
%However,
there is still flexibility in managing fluid volumes. For example, 
a half of the result from $O_8$ can be disposed of before $O_9$ or after $O_9$.
In synthesizing a biochip, these implicit volume management requirements 
as well as the optimal time to deal with them
should be identified by the synthesis tool automatically.
Thereafter, additional flow paths should be constructed to move 
superfluous volumes to the waste port, such as the waste path shown in the
upper left corner in
\figname~\ref{fig:volume_washing}.

Another implicit requirement in biochip synthesis is washing.
During the execution of a bioassay, after some channel segments are 
used either by normal transportation paths or
waste paths, they should be washed before they can be used for normal fluid
transportation again. 
However, washing operations can be postponed if there are still enough 
clean channel segments to construct new transportation paths. 
%In performing washing operations, washing paths can be 
%constructed from clean and used channel segments together.
Consequently,
channel segments can be combined into as few washing tasks as possible 
to improve washing efficiency.

%The postponed washing mechanism can improve the washing efficientcy
%significantly. 
Unlike flow paths that transport fluid samples and reagents,
washing can be performed blockwise instead of pathwise. For example, at
a given moment, the valves on some switches can be closed 
simultaneously to form a wall, as shown by the red dashed line in \figname~\ref{fig:volume_washing}.
Afterwards, the left half of the biochip can be flushed at the same time. 
Combined 
with postponed washing operations,
this technique can increase washing efficiency significantly by 
collecting used paths together and washing them as a whole, instead of performing
traditional pathwise washing operations. 
Furthermore, fluid disposal can be combined with 
washing
operations, 
%as in \figname~\ref{fig:volume_washing}, 
because superfluous fluid to be disposed of and washing fluid all go 
to the waste finally. Consequently, a global 
strategy should be developed to schedule the waste 
paths together
with the introduced block-based washing, or flushing, 
%technique 
to reduce
interference between these auxiliary functions and the normal 
bioassay operations.


\subsectitle{Flow-layer and control-layer codesign}


Valves in a biochip are controlled by air pressure through 
control channels. By switching between a high air pressure and 
a low air pressure in a control channel, a valve can be
closed or opened. In a large design, there may be many valves. If for each
valve an independent control channel is constructed, the routing complexity
becomes very demanding. In addition, for each independent control channel, an
air pressure source 
has to be provided, leading to an
%tremendous 
increase of
cost. %of biochips.

To reduce the number of independent valves, the patterns of valve switching
can be synchronized. If two valves always share the same switching pattern, they can
be connected to the same control channel. In real designs, 
%this
%synchronization may happen, but not very often, because 
this synchronization is very often not possible, because 
%%strictly if there is only one mismatch in the switching patterns of two valves, 
even with a single mismatch in the switching patterns of two valves, 
their control channels
cannot be merged. To alleviate this problem, the switching patterns can be
aligned by examining each 
mismatching switching of two valves.
If the state of a valve can be set to the state of the other valve
without affecting the execution of the bioassay, this mismatching can then be
resolved. 

Since valve switching patterns are determined by the operations and
transportation tasks, a codesign between flow layer and control
layer should be performed. For example,  
there are three valves inside a mixer to mix the fluid samples by peristalsis,
as shown in \figname~\ref{fig:valve_mixer_chip}(b). 
These valves are also shown in \figname~\ref{fig:control_layer}(a), 
which is a
part of the biochip in %the two mixers.
\figname~\ref{fig:biochip_synthesis}(b). 
If the three valves in each of these two mixers always switch with the same 
pattern, they can share the same set of control channels. 
In reality, this is a rare case since the operations executed by these
mixers are usually not aligned in the schedule as shown in
\figname~\ref{fig:schedule_storage_p}(c), where the valves in Mixer1 continue
to switch after Mixer2 finishes 
the execution of the operation $O_3$.

%
\begin{wrapfigure}[20]{r}{0.38\textwidth}
{
  %\revspace{\vskip -2pt}
\figurefontsize
\centering
\input{Fig/control_layer.pdf_tex}
\caption{Flow- and control-layer cosynthesis.}
\label{fig:control_layer}
}
\end{wrapfigure}
%
To improve the alignment of the valve switching activities, 
operations can either be paused or their durations can be  
extended provided that this extension does not affect the results. 
%of the operation.
For example, the 
schedule in
\figname~\ref{fig:schedule_storage_p}(c) can be revised to create 
a new schedule shown in \figname~\ref{fig:control_layer}(b). In this
case, $O_1$ is paused when 
the result of $O_3$ is moved out of Mixer2. 
When $O_2$ starts, $O_1$ resumes its execution.
The execution of $O_2$ is also extended so that 
its ending time is aligned with the ending time of $O_1$.
Afterwards, $O_4$ is delayed 
to align its starting time with the starting time
of $O_5$. With these changes, the switching patterns of valves in the two mixers are
completely aligned so that they
can share the same set of control channels.

In flow- and control-layer codesign, the overall execution time of the bioassay
may be prolonged as illustrated in \figname~\ref{fig:control_layer}(b).
Consequently, in revising the schedule of a bioassay, an objective 
of optimization is to
reduce the extension of the overall execution time as much as possible, by
carefully scheduling the operations to devices and arranging their execution
sequences.

\subsubsection{Macrosynthesis of microfluidic biochips}


%
Biochips considered in microsynthesis above are relatively small-scale.
These chips are very useful in small biochemical labs or in the scenarios such
as point-of-care testing. With the advance of manufacturing technology, 
the integration of biochips has increased tremendously.
Recent advances in manufacturing technologies have enabled a
valve density reaching 1 million per cm$^2$ %\si{\square\centi\metre}
\cite{C2LC40258K}, and consequently,
fully programmable valve arrays (FPVAs) have emerged
for more flexible and highly reconfigurable flow-based biochips
%attempted in the early work
%Quake et. al 
\cite{JMSQ07,matrix11}.
%and the first prototype toward large-scale integration has
%been demonstrated in the work Maerkl et. al 

\begin{wrapfigure}[27]{R}{0.5\textwidth}
{
\vskip -10pt
\figurefontsize
\centering
\input{Fig/fpva_arch.pdf_tex}
\caption{Fully programmable valve array (FPVA).
(a) Architecture \cite{matrix11}. (b)/(c) A 4$\times$2/2$\times$4 dynamic
mixer.
(d) Dynamic mixers of different orientations sharing the same area.}
\label{fig:archi}
}
\end{wrapfigure}

A part of the large valve array in \cite{matrix11} is shown in 
\figname~\ref{fig:archi}(a) to demonstrate the architecture of FPVA, and
videos of fluid transportation and mixing on such a chip can be found from 
\cite{fpva2,fpva3}.
%and a drawing of partial enlargement of four valves
%controlling the four directions of the fluid sample in an enclosed cell at a
%crossing point is shown in \figname~\ref{fig:archi}b.  
In this architecture,
valves (solid blocks) are arranged in a regular structure 
along horizontal and vertical
flow channels (light color). These valves are controlled by air pressure
sources through control channels (narrow channels). 
Similar to the connection grid in \figname~\ref{fig:switch_grid}(f),
transportation paths can be formed by opening and closing specific valves 
on the array, respectively. In addition, the channels also function as
temporary storage caches. The difference between this valve array and the
connection grid is that the former is manufactured directly in this regular
structure to increase integration scale, while the latter is only used to
synthesize a small-scale biochip architecture.

%by opening two valves and closing the other two at an intersection of flow channels like 
%flow paths
%the fluid sample stored there %at the crossing point 
%can be directed to a target location.  
%
%the intended direction %for transportation 
%by forming temporary transportation channels.
%Consequently, flexible flow
%paths can be formed by opening and closing a set of valves, as shown in
%\figname~\ref{fig:archi}c. 

%For example, even a biochip designed in 2008 has contained
%25K valves \cite{JMPK08}. This large-scale system integration enables
%laboratories to execute a large number of biochemical assays on a chip
%simultaneously, opening the door for the long-aspired exhaustive diagnoses 
%to identify illness automatically in hospitals.

Besides transportation channels, mixers can also be constructed on 
the valve array directly, taking advantage of the flexibility and reconfigurability
of such biochips.
%\cite{TsengLHS15}.
For example, a 4$\times$2 mixer and a 2$\times$4 mixer can be constructed as
in \figname~\ref{fig:archi}(b) and \ref{fig:archi}(c),
respectively. In such a dynamic mixer, the eight valves along the enclosed
channel function as peristalsis valves, which switch in a given pattern 
to drive the fluid samples and reagents inside the channel for mixing, similar
to the three valves in \figname~\ref{fig:valve_mixer_chip}(b).
Compared with the traditional mixer in \figname~\ref{fig:valve_mixer_chip}(b), these
dynamic mixers have a different shape and more peristalsis valves, eight in each case, to
form a strong circular mixing flow. % mix different samples.
Moreover, the two mixers in \figname~\ref{fig:archi}(b) and \ref{fig:archi}(c)
can share the same area on the biochip as shown in \figname~\ref{fig:archi}(d),
provided that they are not used at the same time. 

In short, a given area of the valve array can execute various functions such 
as mixing and flow transportation, as well as detection if the corresponding sensors are
included in the area. This flexibility provides a great potential in 
executing large-scale bioassays efficiently. 


%\subsectitle{Hierarchical synthesis of large-scale biochemical assays}
\label{sec:hier_syn}

%During macrosynthesis of biochips, a single valve is not as important as in
%microsynthesis any longer, because of the huge number of valves 
%available. 


%Due to the large number of operations in the bioassay and the
%plentiful resources in the biochip, 
%%To determine the schedule of operations in macrosynthesis, 
%the allocation of devices such as mixers, storage, and transportation channels thus needs to
%be performed at module level instead of being
%processed as a whole % sequencing graph
%%of an application 
%directly. 
Macrosynthesis processes a bioassay with
a large number of operations for resource-intensive
applications. %e.g., exhaustive diagnoses for illness detection.
The biochip executing such a bioassay also contains
plentiful devices such as mixers.
Due to 
%
\begin{wrapfigure}[25]{r}{0.46\textwidth}
{
%  \revspace{\vskip -10pt}
\vskip -3pt
\figurefontsize
\centering
\input{Fig/macro_syn.pdf_tex}
%\caption{Scheduling and synthesis at the macrolevel.} 
\caption{Synthesis at the macrolevel.} 
\label{fig:macro_syn}
}
\end{wrapfigure}
%
%Due to the large number of operations %in the bioassay 
%and the plentiful resources in the biochip, 
the scalability challenges,
it is very difficult to perform microsynthesis 
on an entire bioassay 
of this scale and complexity. Instead, the bioassay is
partitioned into functional subgroups 
%Therefore,
%we partition the bioassay %the sequencing graph 
%into subgroups 
according to the 
relation
between operations. For example, operations between which much fluid transportation
%is required 
exists should be grouped together. %as much as possible. 
Thereafter, the function subgroups are considered as basic modules and mapped
to the biochip as a floorplan. 
After this mapping, microsynthesis as introduced above is performed inside each
subgroup to
improve the execution efficiency of the operations. 
The basic concept of this hierarchical partitioning and synthesis is illustrated in
\figname~\ref{fig:macro_syn}.


The objectives of synthesis at this level include: 1) operations in the
sequencing graph should be partitioned in a way to reduce the communication between
function subgroups; 2) the size of each subgroup should be balanced to reduce the
complexity of microsynthesis inside each subgroup; 3) the floorplan %for the functions 
of the biochip should determine locations of the subgroups, and subgroups requiring
fluid transportation should 
%not be blocked by other subgroups completely; 
be located in close proximity;
4) the floorplan
should consider special devices such as sensors and heaters.
%built in the
%chip for each subgroup. 

During macrosynthesis, the reliability of the biochip should also be
considered. As demonstrated in the video \cite{mixing_store}, 
the three valves in the mixer switch %nearly ten times more often 
much more often
to drive the circular flow inside the mixer than the other valves 
that only control fluid transportation.
Consequently, these valves wear out more quickly than the others.  
To maintain a reliable function of the biochip, operations should be
distributed evenly among devices of the same type to lower the 
worst-case wearout. This reliability issue should be taken into account as
early as during hierarchical partitioning to balance the mixing operations in the
subgroups so that the overall reliability of the biochip can be improved.

%requiring a systematic design flow with a good tradeoff between   
%optimization efficiency and quality.

%\clearpage
\subsubsection{Cyber-physical integration of microfluidic biochips}

In the final phase of this project, the algorithms 
developed in the previous
phases will be
evaluated in the context of a test platform as illustrated in
\figname~\ref{fig:cyber}. 
\revadd{On this platform, sensors provide information about
real-time status of fluid samples, e.g., locations and reaction stages,
so that even 
operations with nondeterministic duration %execution period 
can be executed efficiently, such as that in \cite{sensor_paper} from the
collaborators of this project.} The movement of
fluid samples can also be monitored by a camera and 
%
\begin{wrapfigure}[17]{r}{0.53\textwidth}
{
%\vskip 3pt
\figurefontsize
\centering
\input{Fig/cyber.pdf_tex}
\caption{Integration of a synthesized biochip in a cyber-physical system.} 
\label{fig:cyber}
}
\end{wrapfigure}
%
coarse-grained 
pattern
recognition can be performed by the microcontroller to capture 
%regular
snapshots of the on-going execution. In addition, this cyber-physical
integration provides information about the health of the system, such as
leakage in channels due to manufacturing defects and aging. 
Once known, these defects can be 
circumvented 
by adjusting the schedule of the operations and the mapping of function
subgroups, so that the chip can still work properly. 
%The microcontroller 
%according to the information captured by
%the sensors and the camera, so that faults due to aging can also be
%circumvented during the execution of bioassays. 
%In brief, this final 
\revadd{In this phase, representative assays such as PCR (Polymerase Chain Reaction), IVD (In-Vitro Diagnostics) and CPA (Colorimetric Protein Assay) will be used to 
verify the integration of 
the overall framework from microsynthesis to macrosynthesis 
developed in this project. %and explores new challenges 
%in design and synthesis of microfluidic biochips. 
}


%Due to the promising perspective of biochips, we have successfully obtained
%the support of the project FLUIDA by the International Graduate School of Science and
%Engineering (IGSSE) at TUM. This project focuses on the exporation of the
%variaty in biochip components and devices and the development a system layer
%to provide uses a stable application interface. 
%In addition, together with our visiting professor Tsung-Yi Ho from National Tsing Hua university,
%we have also received the grant from the Institute for Advanced Studay (IAS) to
%support our research on cyber-physical integration of microfludic biochips. 
%The project decribed in this proposal and these two projects
%together form a complete solution for design automation of microfluidic
%biochips.

\vskip 20pt
\subsection{Work programme incl. proposed research methods}

The work plan of this project is divided into six packages (WP1--WP6),
where WP1--WP4 belong to the microsynthesis phase, WP5 belongs to the
macrosynthesis phase, and WP6 belongs to the cyber-physical integration phase.
The work plan is shown in \figname~\ref{fig:work_plan}, where "Stud." means student research
assistant and "BA/MA" means bachelor/master thesis.


\begin{figure*}[h]
%\begin{wrapfigure}{R}{\textwidth}
{
\figurefontsize
\centering
\input{Fig/work_plan.pdf_tex}
\caption{Work plan and work packages.} 
\label{fig:work_plan}
}
%\end{wrapfigure}
\end{figure*}

The relation of these work packages is also shown in
\figname~\ref{fig:task_relation} already. At the lowest level, storage minimization and
architectural synthesis generate a complete biochip architecture with
distributed channel storage. 
%We expect the publish the results in the first conference paper. 
Afterwards, the results of volume management, 
washing, and flow-/control-layer codesign at the second level 
will be overlaid on the generated chip architecture %generated previously 
to form a systematic solution for microsynthesis of biochips. 
%and we expect to publish the result in the second conference paper. 
The tasks of hierarchical design and reliability
optimization deal with large-scale biochip systems, where the
tools from microsynthesis are used to synthesize submodules in
macrosynthesis. Finally, the integration of all the methods is tested in the
cyber-physical integration phase.
%together with the systems developed in our
%partner institutes. Another important task in this integration phase is 
%to explore further challenges in biochip design
%and strengthen the collaboration 
%between the EDA community and the microfluidics community in 
%in this interdisciplinary field.


\subsubsection{WP1: Storage minimization during scheduling}
In the first work package, storage requirements should be reduced during
scheduling. Since the lifespan of a storage is determined by the
ending time of a parent operation and the starting time of a child operation,
as illustrated in \figname~\ref{fig:schedule_storage_p}, storage minimization
can be achieved by solving the optimization problem
(\ref{eq:min_storage_obj}). The challenge of this task is that the minimization
of the overall storage requirements does not guarantee that simultaneously
existing storage requirements are the minimum, which in fact 
determines the required storage capacity that should be maintained in the
chip. In this work package, the optimization problem 
(\ref{eq:min_storage_obj}) will be extended to take this objective into 
account. In addition, heuristic techniques such as extending the list-based
scheduling algorithm in \cite{Micheli94} will be explored to improve the
execution efficiency. 

This work package also includes literature exploration and the development of
a basic framework for the project.

\subsubsection{WP2: Architectural synthesis with distributed channel storage}

With the result of scheduling, a biochip architecture can be generated 
automatically. In the second work package, a new
architecture with distributed channel storage will be explored. This
architecture unifies storage and transportation into a network of channel
segments. Since each channel segment can switch its role between temporary storage 
and transportation, the efficiency of both storage and transportation can be
improved.

This architecture is completely different from the classical von Neumann
architecture as in \figname~\ref{fig:biochip_arch}, 
so that a new model describing the new system behavior needs to
be developed in this work package. To create this model, the constraints for, 1)
device allocation on the connection grid shown in \figname~\ref{fig:switch_grid}, 2) construction of flow
transportation paths, and 3) construction of temporary distributed channel 
storage without contamination, should be established. 
%In addition, all transportation paths and channel storage should 
%not intersect at any moment to avoid
%contamination.

With the constraints above, a framework for automatic architectural synthesis
will be developed. In the first step, we 
%plan to 
formulate this
synthesis task into an Integer Linear Programming (ILP) problem and 
find an optimal chip architecture using an ILP solver. 
This formulation will be improved with heuristic techniques
such as preconfigured device locations and nonregular connection grid
structures
as well as hierarchical modeling to increase the scalability 
in the second step of this work package. 

At the end of this work package, the first milestone of this
project will be achieved, resulting in an efficient biochip architecture 
with distributed channel storage. 
We expect to publish the first paper from this project after finishing this work
package.

\subsubsection{WP3: Volume management and washing}

In the third work package, the challenges of volume management and washing will be
investigated. These challenges are unique for biochips as described in 
Section~\ref{sec:volume_washing}. Both challenges can be formulated as
to construct additional flow paths at given moments.
%, for the purpose of 
%either disposing of superfluous volumes to the waste or conducting 
%cleaning fluid residue in the channels that have been used. 
%Consequently, 
In this work package, we will develop a unified model to deal with these challenges
simultaneously. As illustrated in 
\figname~\ref{fig:volume_washing}, the efficiency of volume management and 
washing can be improved by flushing an area of the chip instead of
individual paths. We will investigate how the valve wall can be constructed 
and how the subtasks of volume management and washing can be grouped together 
for flushing in this work package.

\subsubsection{WP4: Flow-layer and control-layer codesign}

When the control layer in a biochip is considered, switching activities of
valves in devices and in the transportation network should be aligned as much
as possible to reduce the number of independent control channels. This
alignment needs to be considered during both scheduling and architectural
synthesis, as well as in volume management and washing. 
In the scheduling, operations can actually be arranged to align
their starting times and they can also be paused or extended
%operations with long durations can be paused 
so that
the switching activities of valves can be aligned.
During architectural synthesis, flow samples are transported and 
stored in channel segments, so that 
%the open/close states of valves in the transportation network should also be
%considered to reduce mismatch in valve activities. 
they should also be considered to reduce mismatch in valve activities inside
the transportation network.
Furthermore, both volume management and 
washing need to open and close valves. These valve activities, however, have much
flexibility because volume management and washing can be postponed if the ongoing
operations are not affected.

The flow-layer and control-layer codesign can be considered as a fine-tuning
step of the methods developed in WP1--WP3. The challenge is that the
overall execution time of the bioassay should still meet the given specification
after this fine-tuning. 

At the end of this work package, a complete work flow for microsynthesis
of flow-based biochips will be implemented. The result will be summarized and
published as the second paper from this project.


\subsubsection{WP5: Hierarchical synthesis of large-scale biochemical assays} 

While the increasing integration of biochips provides the computing power for
large-scale biochemical analysis, such as future exhaustive tests and
diagnoses in medical care, macrosynthesis of bioassays with a very large number 
of operations needs to be investigated closely.   
Due to their large number, operations in these bioassays cannot be considered
individually during the synthesis process.

In the fifth work package, we will explore methods to partition large bioassays
into function subgroups to reduce the synthesis complexity.
%each subgroup can be and map them to dynamic devices in 
%the FPVA architecture discussed in Section~\ref{sec:hier_syn}. 
This partitioning
and mapping should minimize the transportation requirements between 
function subgroups. In mapping these
function subgroups to a biochip, the floorplan should guarantee that the 
subgroups with much transportation are allocated not
far away from each other to avoid unnecessarily long transportation 
channels. %and potential storage.

For each subgroup, the microsynthesis flow developed in the first phase of this
project will be used to optimize the internal structure 
%of each functional group on the FPVA 
to improve execution efficiency.  
We plan to publish the result of this work package as the third paper from
this project.

%
\subsubsection{WP6: Reliability-aware macrosynthesis}

In a biochip, the valves in a mixer for peristalsis switch far more often
than the other valves controlling flow transportation. 
Consequently, these valves wear out quickly, leading to reliability issues 
in executing large-scale biochemical assays. 

This reliability problem can be alleviated by distributing the function of
peristalsis to different valves
%. By rotating the valves 
so that the switching
activities on different valves can be balanced. As shown in
\figname~\ref{fig:archi}(d), the valves inside the dynamic mixers do
not overlap, although they occupy the same area on the FPVA. We will
take advantage of this feature to map dynamic mixers with different
orientations and in different areas to reduce the maximum valve usage. 
This optimization needs to be considered in both hierarchical partition 
and floorplan on the FPVA. 

After finishing this work package, the results of dynamic mapping and
reliability improvement will be published as the third paper from this project.


\subsubsection{WP6: Cyber-physical integration of microfluidic biochips}

In practice, a biochip system may contain sensors and cameras to monitor the
movement of fluid samples. In addition, applications may include
operations with flexible durations. For example, some operations need to be stopped
when the reaction reaches a given threshold. These information can be
collected by sensors and cameras and sent to the microcontroller to adapt the
valve switching patterns. 

\revadd{In the last work package, the algorithms developed in the previous work packages
will be integrated and tested together with the collaborators. 
%with our partner institutes, from which cyber-physical biochip systems with adaptive resynthesis and
%reconfiguration are available. 
The data from sensors in the platform of the collaborators
%and cameras from the work 
\cite{sensor_paper}
will be analyzed and used to improve the synthesis flow introduced in this
project.}
In addition, the chance to expand this work flow to incorporate further
features and deal with challenges emerging from this system integration will
be explored.

%the locations of sensors on a biochip will be
%investigated monitor the status of bioassay execution efficienty with only a few
%%sensors. The images captured by the camera will be used assist this dynamic
%controlling mechnism. During the execution of an bioassay, once the the progress
%deviated from an assumed aaa, the following execution of the bioassay will be
%resynthesis online from a set given template to maintain the quality of the
%results.

The last work package integrates the methods developed in WP1--WP5 to form a
systematic solution for synthesis of flow-based biochips. The results of this
project will be summarized and published as a journal article. 
%at the end of this work package.
%such as at
% {{IEEE} Trans. Comput.-Aided Design Integr.  Circuits Syst. (TCAD)}.

\vskip 10pt
\subsection{Data handling}

The results of this project will be published at leading conferences and
journals in the field of design automation for integrated systems. 
%These
%papers will be uploaded to the publication portals such as the IEEE Xplore Digital
%Library and the ACM Digital Library to maintain its exposure to researchers.

To facilitate the access to these papers, they will also be published with the IEEE open
access publication option, so that researchers around the world 
have free access to them.
This option will enhance the reuse of the
methods developed in this project and improve the interdisciplinary research
on design automation for emerging systems.


\subsection{Other information}

None


\subsection{Explanations on the proposed investigations}

This project does not involve humans and human materials as well as animals in
the experiments.

\vskip 10pt
\subsection{Information on scientific and financial involvement of
international cooperation partners}

%In this project, the Institute for Electronic Design Automation (EDA) at TUM
%will collaborate with the research group of Prof. Krishnendu Chakrabarty from
%Duke University, USA, and the research group of Prof. Tsung-Yi Ho 
%in National Tsing Hua University, Taiwan. 
%Prof. Chakrabarty is  a leader in the field of microfluidic biochips and 
%microbiology applications. His research ranges from application 
%mapping to manufacturing testing.
%His expertise on design, test and microfabrication will provide us an
%extensive support during this project.
%Prof. Tsung-Yi Ho is a leading scientist in microfluidic integration.
%His research focuses on cyber-physical integration of biochips, which
%will be a very important input to this project.
%In addition, we will work actively with several other 
%international researchers on modeling and analysis of reliability effects,
%e.g., research groups at CMU, USA, and NTU, Singapore. 


In this project, %the Institute for Electronic Design Automation (EDA) at TUM
we will collaborate with the research group of Prof. Krishnendu Chakrabarty from
Duke University, US, and the research group of Prof. Tsung-Yi Ho from
National Tsing Hua University, Taiwan. 

\revadd{Prof. Chakrabarty is one of the pioneers to apply computer
  algorithms to microfluidic design. 
His research ranges from application 
mapping to manufacturing testing, which has resulted in more 
than 550 papers published at top conferences and journals, in which 
  more than 250 papers are on design automation for microfluidics
  (\url{http://people.ee.duke.edu/~krish/pubs.html}).
 He has received many highly recognized best paper awards on this topic, e.g.,
the IEEE Transactions on
Computer-Aided Design of Integrated Circuits and Systems
Donald O. Pederson Best Paper Award in 2015.
%He has served as the Editor-in-Chief of 
%ACM Journal on Emerging Technologies in Computing Systems,
%2010--2012 and IEEE Design \& Test of Computers, 2010-2012, and 
%He is serving as
%the Editor-in-Chief of the IEEE Transactions on Very Large 
%Scale Integration (VLSI) Systems. 
He is a Fellow of IEEE and a Fellow of ACM.
%His expertise on design, test and microfabrication will provide us an
%extensive support during this project.
}

\revadd{Prof. Tsung-Yi Ho is a leading scientist in microfluidic integration and has
published more than 150 papers on this topic 
(\url{http://www.cs.nthu.edu.tw/~tyho/publications.html}). 
He has been the recipient of the Humboldt Research
Fellowship (2012--2013), which he spent in TUM. He has
also received the Hans Fischer Fellowship from TUM-IAS.
%His research on cyber-physical integration of biochips 
%received the Best Paper Award at the VLSI Test Symposium (VTS)
%in 2013. 
%Currently he is serving as an ACM Distinguished Speaker, 
%Associate Editor of the ACM Journal on Emerging Technologies in Computing
%Systems and the IEEE Transactions on Computer-Aided Design of Integrated
%Circuits and Systems.
%His experience in system integration of microfluidic biochips will be a very
%important input to this project.
}

\revadd{%In this collaboration, 
  %the project of this proposal focuses on architectural synthesis of microfludic
  %biochips from the function perspective. Prof. Chakrabarty will analyze new
  %architectures from the test view to evaluate the reliability and testability
  %of the new architectures. Prof. Ho will provide support for design rules we
  %use in generating new biochip architectures. Both researchers have been
  %working on joint projects to manufacture biochips to test the design
  %automation algorithms, for example, in their current MEDA project in Taiwan,
  %which provides precious support to the proposed project to clarify design
  %rules for the design flow of micrfludic biochips.
In the collaboration, I will focus on architectural synthesis of microfluidic
  biochips from the functional perspective. Prof. Chakrabarty will analyze new
  architectures from the test view to evaluate their reliability and
  testability.
  Prof. Ho will provide support for design rules for the
  automated design flow.
  %Both researchers have been
  %working on joint projects to manufacture biochips to test the design
  %automation algorithms, for example, in their current MEDA project in Taiwan.
  %Meanwhile, 
  Both professors have visited TUM with funding support from 
  the Alexander-von-Humboldt-Stiftung and from the 
  Institute for Advanced Study at TUM. They will continue to 
visit us in the coming years to strengthen our collaboration.
}

  %Both Prof. Chakrabarty and Prof. Tsung-Yi Ho
  %professors have visited TUM with with funding support from the Alexander-von-Humboldt-Stiftung 
  %and from the Institute for Advanced Study at TUM. They will continue to 
  %visit us in the coming years to strengthen our collaboration. Therefore, 
%they will not be financially involved into the proposed project.
%Technical meetings will be arranged regularly at conferences 
%and their potential visits to TUM will be funded by other sources.

\revadd{During the period of this project, knowledge exchange with researchers in the
microfluidics community, e.g., Prof. Dr. Oliver Hayden in the Department of
Electrical and Computer Engineering, TUM,
who is working on microfluidics extensively covering
characterization, diagnostic, and treatment platform technologies
in biomedical science,
%, and Prof. Dr. Christoph Kutter from
%Fraunhofer EMFT in Munich, 
will be conducted often 
%to help the advance of this project. From Prof. 
%Hayden, 
to obtain 
information of microfluidic applications and feedback on the proposed
framework.
}





