\begin{abstract}

Flow-based microfluidic biochips have attracted much attention in the EDA community due to their miniaturized size and execution efficiency. Previous research, however, still follows the traditional
computing model with a dedicated storage unit, which actually becomes a bottleneck of the performance of biochips. In this paper, we propose a distributed channel-storage architecture (DCSA) to cache fluid samples inside flow channels temporarily.
Since distributed storage can be accessed more efficiently than a dedicated storage unit and channels can switch between the roles of transportation and storage easily, biochips with this architecture can achieve a higher execution efficiency even with fewer resources. Furthermore, we also address the flow-path planning that enables the manipulation of actual fluid transportation/caching on a chip. Simulation results confirm that the execution efficiency of a bioassay can be improved significantly, while the number of valves in the biochip can be reduced accordingly. Also, flow paths for transportation tasks can be constructed and planned automatically with minimum extra resources.

\begin{IEEEkeywords}
Microfluidic biochips, channel storage, scheduling, architectural synthesis, flow-path mapping.
\end{IEEEkeywords}

\end{abstract}
