15 gid=3000000
15 uid=3603518
25 ctime=1538382170.5264
27 atime=1538916861.459084

\section{Synthesis of Biochips Considering Storage and Caching}
\label{sec:storage_synthesis}

Storage and caching should be considered from scheduling to architectural synthesis
to reduce storage requirements and to construct channels for distributed
storage.
Correspondingly, the proposed method includes three major parts:
1) intermediate fluid storage is minimized in scheduling and binding,
2) all the allocated devices are assigned to exact locations on a biochip, and 3) channel segments that cache intermediate fluid samples are constructed among devices, leading to a
distributed channel-storage system which fulfills the tasks of transportation and storage at the same time.

\subsection{Minimizing Storage in Scheduling and Binding}
\label{sec:scheduling_binding}

Scheduling and binding assign operations in a given sequencing graph
to time slots of specific devices.
%This is a common step in high-level synthesis of
%integrated circuits and has been discussed in detail in
%\cite{Micheli94}. %Since we
To optimize storage requirements,
%when executing the assay,
we formulate the scheduling and binding task as
an ILP problem. %instead of the commonly used List algorithm.

The variables and constraints for scheduling and binding are listed in
Table~\ref{tb:vc_scheduling_binding}. The uniqueness constraint (\ref{eq:uniqueness})
specifies that operation $o_i$ should be assigned
to one device only. The duration constraint ensures
%that operation
$o_i$ has enough time to finish.
The precedence constraint (\ref{eq:precedence}) guarantees that a child
operation must be executed later than its parents. Finally, the
non-overlapping constraint (\ref{eq:non_overlapping}) prevents two operations
whose operation periods overlap from being executed by the same device.
These constraints are common for high-level synthesis and
widely used in synthesis methods for biochips \cite{SuCh04}.

To minimize the execution time of the assay, another variable $t^E$ is
used to represent
the latest ending time of all operations, constrained as
\begin{equation}\label{eq:latest_finishing}
t^e_i \le t^E,  \quad \forall o_i\in O.
\end{equation}
\vskip 7pt
By minimizing $t^E$, the operations in the sequencing graph
are assigned to proper time slots %as early as possible
to produce a compact schedule.

% as in \figname~\ref{fig:pcr}(b) and \ref{fig:pcr}(c).
%These two schedules
%%in \figname~\ref{fig:pcr}(b) and \ref{fig:pcr}(c),
%however, require different storage access and capacity.
%Consequently, they lead to different execution times of the bioassay
%eventually.

%Model transportation starting and ending time?

To reduce storage requirements, we introduce an additional \textit{storage minimization
objective}.
\textcolor{red}{Assume that the pure transportation time from a device to another device
is a constant $u_c$.}
If the schedule produces a transportation time larger
than $u_c$, the fluid sample must be cached somewhere before it is
used ($u_c$ is set to 1 in our experiments).
Fig.~\ref{fig:schedule_storage} illustrates
the schedules of executing an assay with five operations by two devices. The sequencing graph of this assay is shown in Fig.~\ref{fig:schedule_storage}(a).
In \figname~\ref{fig:schedule_storage}(b),
operation $o_2$ is scheduled before $o_3$. Consequently, the result of $o_2$
must be stored until it is used by $o_4$ and $o_5$, leading to two storage
requirements. %lasting six time units in total.
In \figname~\ref{fig:schedule_storage}(c), $o_3$ is scheduled before $o_2$,
leading to only one storage requirement lasting a shorter time.
%four time units.
%Although these two schedules lead to the same execution time for the assay,
%the additional storage requirements in \figname~\ref{fig:schedule_storage}(b)
In this example, we can observe that
%should be met by more resource built in the chip.
the lifetime of stored fluid samples is determined by the difference between
the ending time of the parent operation and the starting time of the child
operation $u_{i,j}$ defined in Table~\ref{tb:vc_scheduling_binding}. Consequently, the total storage requirement can be reduced by
solving the following ILP problem:
\begin{align}
\text{minimize} & \quad \alpha t^E+\beta\sum_{{(o_i,o_j)\in E\bigwedge d_i\ne d_j }}u_{i,j}
 \label{eq:minobj}\\
\text{subject to} & \quad
\text{(\ref{eq:uniqueness})--(\ref{eq:latest_finishing}})\label{eq:op_cond}
\end{align}
%where $\sum_{\substack{(o_i,o_j)\in O\\d_i\ne d_j  }}(u_{i,j}-u_c)$ is the total storage
%time in the schedule.
where $\alpha$ and $\beta$ are constants to control the priority of
execution time and storage requirement minimization.
$d_i \ne d_j $ excludes the operation pairs assigned to the same device
so that they do not need transportation.
%Note that the formulation above
%overestimates the storage requirement of $o_2\to o_5$ in
%\figname~\ref{fig:schedule_storage}(b). This difference is
%modeled in our method with another variable and the solver determines which
%fluid sample should be transported first.

\begin{figure}[t]
{\figurefontsize
\centering
\input{Fig/schedule_storage.pdf_tex}
\caption{Storage reduction. (a) Sequencing graph. (b) Schedule with two
storage requirements. (c) Schedule with one storage requirement. The execution
times of the assay with these two schedules are equal. }
\label{fig:schedule_storage}
}
%\vspace{-0.5cm}
\end{figure}

%where all operation are assigned to
%given devices and their starting and end times as well as the time slots for
%storage are determined


%!!!! CAnother issue in the storage-aware scheduling is the
%order of the result of one operation
%going to two devices. For example, the result of operation $o_2$ goes to $o_4$
%and $o_5$. The precedence of these two transportation
%is modeled with a 0-1 variable. Similarly, the order of two inputs entering
%one device is also modeled one such a variable. The solver determines which
%tranportation is performed first to minimize the execution time and storage
%requirement.

\subsection{Architectural Synthesis with Distributed Channel Storage}\label{sec:channel_storage_synthesis}
After solving the
optimization problem (\ref{eq:minobj})--(\ref{eq:op_cond}), we have a schedule
similar to \figname~\ref{fig:schedule_storage}(c),
including the information: 1) to which devices operations are
assigned; 2) the starting time and the ending time of each operation; 3) the
%existing periods
starting time and the ending time of each fluid storage requirement.
%$of all fluid samples stored in the chip.

The schedule, however, only defines the transportation requirements between
devices after operations are executed. In the chip, physical channels need
be constructed to conduct these transportation tasks.
When a large assay is executed by several devices,
transportation channels need to be built between nearly
any pair of devices to move fluid samples
%between devices
efficiently.
Since the flow-layer in a biochip is only two-dimensional, eventually
intersections between channels cannot be avoided. At an intersection,
a switch should be built to direct the transportation
flow to the target device. A switch is constructed by
four valves at an intersection, as shown in \figname~\ref{fig:switch}(a). At a
given moment, two out of the four valves in a switch are opened to
connect two channel segments. Consequently, a transportation
channel between two devices becomes a path
formed by several channel segments connected by switches.
Such a path is called a \textit{transportation path} henceforth.

Besides channels, the locations of devices should also be determined.
These locations should be assigned together with the construction of
transportation channels, because the distance and relative locations of devices
affect how channels are constructed and how they intersect with each other.

Considering devices and channels together, the architecture of a biochip can
be described as devices surrounded by channel segments in the form of a grid.
%This is different from normal IC design because crossing is allowed in this
%architeture, but the channels crossing the same node should not be used at the
%same time to avoid contamination.
For example, the architecture of a biochip with five devices is
shown in \figname~\ref{fig:switch}(b), where the smaller nodes
represent switches and the larger nodes represent devices.
Transportation paths between devices are
formed from channel segments connected by switches, e.g., path 1 and path 2
in \figname~\ref{fig:switch}(b).
Since transportation paths are used only when there is a fluid sample
traveling along it, channel segments can be reused by
different paths so that the efficiency of channel segments increases.


\begin{figure}[t]
{\figurefontsize
\centering
\input{Fig/switch.pdf_tex}
\caption{Switch and channel storage. Large nodes represent devices and
small nodes represent switches.  (a) Switch structure. (b) Two paths
sharing one channel segment with time multiplexing. (c) Fluid sample to
channel storage. (d) Storage in channel segment. (d) Fluid sample from channel
storage to device.}
\label{fig:switch}
}
%\vspace{-0.5cm}
\end{figure}

With transportation paths formed from channel segments, the proposed
distributed channel storage concept can thus be formulated.
%In addition, the fluid storage need also be implemented in the
%chip.
%As discussed in Section~\ref{sec:motivation},
%overcome the bandwidth limit problem at the ports of the dedicated storage unit
As illustrated in \figname~\ref{fig:device_storage}(c),
%a port of the dedicated storage unit is
%designed in the style of a multiplex. When the storage unit is accessed, the valves
%to one cell is opened to allow the fluid sample to flow in. The other valves are closed to
a dedicated storage unit suffers bandwidth problem at its ports
because multiple fluid samples must be queued when they access the storage unit
at the same time.
Observing that the storage cells
inside a dedicated storage unit
%that store fluid samples
are in fact channel segments, as shown in
\figname~\ref{fig:valve_mixer_storage}(c), we distribute fluid storage
directly in channel segments. For example, in
\figname~\ref{fig:switch}(c), along path 3 a fluid sample is moved to the channel
segment between A and B. However, the next operation using this
fluid sample is scheduled later, so that the fluid sample must stay in the
channel segment. During the lifetime of this storage, the channel segment
between A and B cannot
be used by other paths
and the valves at the two ends of this channel segment should be closed.
Further transportation tasks between devices are, however, still
fulfilled by paths not including this channel segment,
as path 4 and 5 in \figname~\ref{fig:switch}(d).
When the stored fluid sample is finally needed, it is moved to the target
device again by a newly constructed transportation path, shown as path 6
in \figname~\ref{fig:switch}(e).

Unlike the dedicated storage unit shown in
\figname~\ref{fig:valve_mixer_storage}(c), the distributed storage in a
channel segment has a higher access efficiency.
When a fluid sample stays in a channel segment,
that segment is turned into a \textit{storage segment}. When the
fluid sample moves again, the segment becomes a part of
the transportation path. This concept of channel role switching
unifies transportation and storage,
%so that a dedicated storage unit is not required anymore. Consequently, the efficiency
and
%increases channel usage significantly.
%because
the low-efficiency channels forming storage cells
in a dedicated storage unit are excluded completely.
In addition, this on-the-spot caching is closer to the target
device than a dedicated storage unit, so that the execution efficiency of
the assay can also be improved.

\begin{figure}[t]
{\figurefontsize
\centering
\input{Fig/grid_model.pdf_tex}
\caption{Connection grid.}
\label{fig:grid_model}
}
%\vspace{-0.5cm}
\end{figure}

To synthesize the architecture of a biochip from its schedule requires to
determine the relative locations of devices
as well as channel segments and the switches connecting them as shown by the examples
in  \figname~\ref{fig:switch}(b)--(e). The devices, switches and their connections together form a
\textit{connection graph}. %nodes edges}
A valid connection graph should be capable of
constructing all transportation paths specified in the schedule and caching
intermediate fluid samples in channel segments.
To reduce resource usage, the synthesized connection graph
should contain as few edges as possible.

The connection graph is generated using a general \textit{connection grid}, as
shown in \figname~\ref{fig:grid_model}. At each node $n_i$ on the grid,
either a device or a switch can be assigned. An edge $e_j$ represents
a channel segment capable of caching a fluid sample.
%The transportation channels are contructed by the
%channel segments.
%
%At each channel segment, we can cache one fluid sample. To
%determine on which nodes the devices are assigned and to construct the
%transportation channels, we establish a ILP formulation. In solving the
%optimization problem, we minimize the number of edges in the grid used by
%transportation channels, so that only the minimum connections are kept in the
%final result of architectural systhesis.
%
%In the connection grid, a node is denoted by $n_i$ and an edge is denoted
%by $e_j$, where $i$ and $j$ are the node and edge indexes, respectively.
We use a 0-1 variable $a_{i,k}$ to represent whether device $d_k$
is assigned to node $n_i$. Since a node can be occupied by no more than one device
and a device must be assigned once, $a_{i,k}$ can be constrained as
\begin{equation} \label{eq:device_node_1}
\sum_{d_k\in D} a_{i,k}\le 1, \forall n_i\in N, \quad
\sum_{n_i\in N} a_{i,k}=1, \forall d_k\in D
\end{equation}
%+a_{i,k_2}\le 1$ for two devices
where $N$ is the set of nodes in the connection grid and $D$ is the set of
devices.

Assume there is a transportation path $p_r$ between
device $d_{k_1}$ and device $d_{k_2}$ in the period between
$t^s_{r}$ and ${t^e_{r}}$, where $r$ is the index of the path.
$t^s_{r}$ and ${t^e_{r}}$ are constants determined in the
scheduling and binding step in Section~\ref{sec:scheduling_binding}.
We use a 0-1 variable $\epsilon_{j,r}$ to represent whether the edge
$e_j$ is on the path $p_r$.
To construct a path between $d_{k_1}$ and $d_{k_2}$, we need to
guarantee that the path starts from the node for $d_{k_1}$
and ends at the node for $d_{k_2}$.
Consequently, at one of these two nodes, only one of the four edges incident
to the node can be covered in the path.
At each other node on the path, exactly two edges are covered by the path, as
illustrated with path $p_r$ in \figname~\ref{fig:grid_model}.
%For each node $n_i$, we then assign
%another variable $g_{i,r}$ to represent the number of edges incident to $n_i$
%and covered by the path $p_r$.
Accordingly, we can
construct the path with the following constraints
\begin{equation}\label{eq:degree}
\sum_{e_j\in E_i}{\epsilon_{j,r}}\ge 2-a_{i, k_1}-a_{i, k_2}-(1-y_{i,r})M,
\sum_{e_j\in E_i}{\epsilon_{j,r}}\le y_{i,r}M
\end{equation}
where $E_i$ is the set of edges incident to node $n_i$; $y_{i,r}$ is an
auxiliary 0-1 variable to indicate
whether there is an edge on $p_r$ that is incident to $n_i$;
%an edge incident to $n_i$ is covered by the transporation path,
$M$ is a very large constant to transform the two
situations indicated by $y_{i,r}$ into linear constraints
\cite{chen2011applied}.
The path construction constraints become more complex when a storage is involved,
            which leads to three sub-paths: 1) from $d_{k_1}$ to a storage segment; 2)
            the storage segment caching the fluid sample; 3) from the storage segment to the
            target device $d_{k_2}$.
            We denote the three transportation paths as $p_{r, 1}$,
            $p_{r, 2}$ and $p_{r,3}$, as illustrated in \figname~\ref{fig:grid_model}.
            Since the end node of $p_{r,1}$ and the starting
            node of $p_{r, 3}$ can be any node on the connection grid as long as they are
            the two end nodes of the same edge,
            we use 0-1 variable $a_{i_1,r_1}$ to represent that node
            $n_{i_1}$ is the last node on the path $p_{r,1}$ and the
            variable $a_{i_2,r_2}$ to represent that node
            $n_{i_2}$ is the first node on the path $p_{r,2}$.
            Afterwards, we create constraints similar to (\ref{eq:degree}) for each
            sub path. In addition, we include the constraint that $n_{i_1}$ and $n_{i_2}$
            are the two end nodes of the same edge.

In the schedule, there are many transportation paths
%, which should be contructed on the connection graph.
at a given moment. These paths should not
intersect at a node or share the same edge to avoid conflict.
%In our formulation,
Therefore, we examine the paths on the connection grid at the starting time of each
transportation path, because this is the moment a new transportation is
initiated. At each such moment $t$, we constrain that all the paths existing on
the connection grid should not share any edge or intersect at a node, as,
\begin{equation}\label{eq:edge_node_sharing}
\sum_{e_j\in E} \epsilon_{j,r}\le 1, \forall p_r\in P_t, \quad
\sum_{n_i\in N} \eta_{i,r}\le 1, \forall  p_r\in P_t
\end{equation}
where $P_t$ is the set of paths existing at time $t$;
$\eta_{i,r}$ represents whether node $n_i$ is on path $p_r$;
$E$ and $N$ are the sets
of edges and nodes in the connection grid, respectively. Exception of
constraint (\ref{eq:edge_node_sharing}) is that the two ending nodes of the
storage segment $p_{r,2}$ can be passed by other paths when the fluid sample is stored,
as path $p'_r$ in \figname~\ref{fig:grid_model}, so that their variables
$\eta_{i,r}$ are not included in (\ref{eq:edge_node_sharing}) when $p_{r,2}$
exists. %in the grid.

When generating the architecture of the chip, we minimize the number of edges
that are really used by the transportation paths
to reduce resource usage. If an edge is used once by any transportation path,
it should appear in the chip. We use a 0-1 variable $s_j$ to represent whether
a channel segment should be kept in the chip, and constrain it as
\begin{equation}\label {eq:edge_app}
s_j\ge \epsilon_{i,r}, \; \forall p_r\in P
\end{equation}
where $P$ is the set including all transportation paths.

Finally, the architecture of the biochip can be determined by solving the
following optimization problem
\begin{align} \label{eq:edge_opt}
\text{minimize} & \quad \sum_{e_j\in E}s_j\\
\text{subject to} & \quad
\text{(\ref{eq:device_node_1})--(\ref{eq:edge_app})}\label{eq:edge_opt_cond}
\end{align}
After determining $s_j$, the edges and nodes
that are not used in the connection grid
are removed to generate the chip architecture as a planar
connection graph similar to
\figname~\ref{fig:switch}(b)--(e).

%\textcolor{red}{Note that since the architectural synthesis above is performed based on a virtual grid, i.e., the connection grid shown in \figname~\ref{fig:grid_model}, the resulting architecture only defines the relative positions of devices and their connections by switches, i.e., an ``open architecture" for chip designers. To generate the physical design of a biochip from its architecture, the lengths of flow channels can be adjusted flexibly according to the volume requirements of fluid caching/transportation. Assume that the capacity per unit length of a microchannel is $C_u$ (the value of $C_u$ can be varied due to the design specification in industry, such as thickness and diameter of microchannels), if the volume of the fluid that needs to be cached is $C_f$, the length of the channel should be set at least to $C_f/C_u$. }

% \subsection{Iterative compression of chip area}
%
% The chip architecture defines the relative locations of devices and
% their connections by switches. The length of an edge, if used as storage,
% should be large enough to accommodate one fluid sample.
% %Although we can
% %build a chip with the architecture that works with distributed
% %channel storage directly,
% Although we can directly build a biochip with a distributed
% channel storage architecture as shown in
% \figname~\ref{fig:layout}(a)$\to$(b), the size of the chip may be too large because there is
% much unused space between the channel segments.
% %Furthermore, in the architecture, the
% %devices are abstracted as nodes.
% %In reality, their dimentions are quite large,
% %so that  assigning a device to a node directly lead to wire conflict.
%
% To generate the physical design of a biochip from its architecture, we first
% insert the devices to the connection graph. Since devices are relatively large,
% we extend the lengths of channel segments to meet all length requirements for
% channel storage. Afterwards, we collapse the segments toward the upper
% right corner iteratively. In each iteration, we either
% reduce the horizontal dimension or the vertical dimension by
% a constant number and generate bending points on the
% segments to extend segment lengths.
% The iterations stop when the layout cannot be compressed further and the
% result is the final physical design. The major steps of this process are
% illustrated in \figname~\ref{fig:layout}.
%
%
% \begin{figure}[t]
% {\figurefontsize
% \centering
% \input{Fig/layout.pdf_tex}
% \caption{Iterative layout compression. (a) Device assignment. (b) Layout
% expansion. (c) Layout compression.}
% \label{fig:layout}
% }
% \end{figure}


\subsection{\textcolor{red}{Scheduling Recomputation}}\label{sec:schedual_recomputation}
\textcolor{red}{
As discussed previously, since the layout of the biochip has not been determined during the high-level synthesis stage, we use a constant $u_c$ to estimate the durations of transportation tasks. After completing the architectural synthesis above, a biochip architecture that defines the  positions of devices and the corresponding connections among them is generated. With this chip architecture, the execution time of each transportation task in the scheduling can be computed directly. Thereafter, the generated chip architecture is fed back into the high-level synthesis stage by re-performing the proposed scheduling method, thus updating the execution procedure of the complete bioassay with accurate transportation latencies. Note that the velocity of fluid flow is set to 10 mm/s in the our method \cite{minhass2012architectural}.
}










